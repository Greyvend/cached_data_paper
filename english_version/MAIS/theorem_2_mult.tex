%\author{Mosin Sergey, Zykin Sergey}
\begin{theorem}
$P^{\ast} =  \pi_{X^{\ast}} ( \sigma_{F^{\ast}} (P_1 \Join \dots \Join
P_n))$, where $X = \bigcupn X_{v}$ if:
\\a) $X^{\ast} \subseteq X$, $X_v \supseteq \<\Join_{i=1}^{s(v)} R^v_i> \cap (\bigcup\limits_{\substack{w=1\\ w \neq v}}^{n} \<\Join_{i=1}^{s(w)} R^w_i> ), v = 1,\dots,n$
\\b)
$ \bigcupn \{R^{v}_{1}, \ldots, R^{v}_{s(v)}\} = \{R'_{1}, \ldots, R'_{s'}\}
= \{R^{\ast}_{1}, \ldots, R^{\ast}_{l}\} $
\\c) $M(F^{\ast}) \subseteq M(F_{v}), v = 1,\dots,n $
\\d) $ \<F^{\ast}> \subseteq X $.
\label{th_mult_eq}
\end{theorem} 
\begin{proof}
Conditions of the theorem are a special case of theorem \ref{th_mult}, so
inclusion  $P^{\ast} \subseteq \pi_{X^{\ast}} (\Join_{v=1}^{n} P_{v})$ is
considered to be proved. It is necessary to show, that $\pi_{X^{\ast}}
(\sigma_{F^{\ast}} (\Join_{v=1}^{n} P_{v})) \subseteq P^{\ast}$. Assume tuple
$t \in \pi_{X^{\ast}} (\sigma_{F^{\ast}} (\Join_{v=1}^{n} P_{v}))$. Let's show,
that $t \in P^{\ast}$. Denote $\sigma_{F^{\ast}} (\Join_{i=1}^{s(v)} R^{v}_i)$
as $P'_v$, so $P_v = \pi_{X_v} (P'_v)$. Using the property of the projection
operation and condition a), we have $P_1 \Join \dots \Join P_n = \pi_{X_1}
(P'_1) \Join \dots \Join \pi_{X_n} (P'_n) = \pi_{X} (P'_1 \Join \dots \Join
P'_{n})$. Thus, according to the definition of tuple $t$, there is a tuple  $t' \in R'_1
\Join \ldots \Join R'_{s'}$, such that $t = t'[X^{\ast}]$, $F^{\ast} (t')=\n{TRUE}$.
According to the commutativity of natural join, $R'_1 \Join \ldots \Join
R'_s = R^{\ast}_1 \Join \ldots \Join R^{\ast}_l$, hence, $t' \in R^{\ast}_1 \Join
\ldots \Join R^{\ast}_l$. Therefore $t \in P^{\ast}$. The theorem has been proven.
\end{proof} 
