\documentclass[10pt,a4paper]{article}
\author{Mosin Sergey, Zykin Sergey}
\usepackage{amssymb,amsmath}
\usepackage[utf8]{inputenc}
\usepackage{amsthm}
\usepackage{amsfonts}
\usepackage{algorithmic}
\usepackage{algorithm}
\usepackage{xcolor}
\usepackage{textcomp}
\floatname{algorithm}{Algorithm}
\newtheorem{theorem}{Theorem}
\newtheorem{mydef}{Definition}
\newtheorem{statement}{Statement}
\newtheorem*{consequence}{Consequence}
\newtheorem{note}{Note}
\def \eval #1#2{\left.#1\right\vert_{#2}}
\def \<#1> {\langle #1 \rangle}
\def \n #1{\mathit{#1}}
\title{Building table-like applications using cached data}
\author{
Mosin Sergey, Zykin Sergey
\footnote{The paper is supported by RFBR grant \#12-07-00066-a}
}
\begin{document}
\maketitle

\section{Abstract}
We primarily address the usage of cached Database query results in our paper.
This problem was already discussed and has some software implementations. In our
work, we propose Truth spaces usage in logical formulas while querying a Database.
This approach is supposed to help an analyst, who works with dynamically built
data representations logically described as multidimensional tables. Those
representations don't take much disk space, so one can use GPU to speed up the
computations. We consider queries in a form of relational algebra expressions,
similar to "universal relational query". Logical formulas in the queries are
presented in Disjunctive Normal Form. They consist of atomic expressions
consisting of SQL predicates. In the paper we consider single and multiple table
cases to build new data representations.

\section{Introduction}
A lot of current Big Data analysis solutions use Online Analytical Processing
technologies, formed in early 90s \cite{codd}. A lot of fundamental
\cite{lecht,lehner,mazon} and applied \cite{vassi, peder, progressive, giorg}
results were obtained in this field lately.

The purpose of this paper is to study the problem of building Dynamic
Multidimensional Data Representation from the Relational Database using
cached data representations on a user computer, equipped with a Graphical
Processing Unit (GPU). We don't require any transformation of the initial
Database schema, even to the hierarchical model.

This problem is connected with a field of query optimization, because it aims to
decrease the data transfer from a Database server. Cached data is actively
used in Database Management Systems (DBMS), but mostly, it is only repeating use
of data, written in cache, without any prior data analysis to define any partial
or combined use. The DBMS can only avoid requesting blocks of data from external
devices while serving the query, if they are present in the cache. So, the
block numbers are analyzed, not their information.

This paper is based on the results, obtained in \cite{zyk_pol}. We removed
condition that limited Intermediate Data Representation (IDR) attributes and
also made some generalizations on multiple IDRs case.

\section{Existing solutions overview and comparison}

In a field of fundamental researches, huge amounts of publications aim at optimal
query plan generation based on a formal set of rules that don't use SQL
operator predicate domains (logical optimization). In other cases, those domains
are taken into account for statistical evaluations to improve physical access to
the data. Stream data query executing problems, described in \cite{Olston03,
Denny05}, are related to our problem closer, but different aims lead us to
different results. The most similar to our work is paper \cite{Afrati06}. They
analyze conjunctive queries on data domains with predicates in the form of
arithmetical comparisons, and present query computation algorithms using IDRs. In
our paper, the special case of universal relational query is considered. It is a
query on the Database relations, not particular domains. Although we had similar
aims, the results obtained differ because of stated factors. In particular,
there is no need to create any algorithms of data selection in our work, as they
are replaced by Relational Algebra.

There are some works, which examine the query optimization problem in the case of
multidimensional warehouse representation with the use of dynamically formed
representations. Two basic methods are used there: statical \cite{baralis,
gupta, gupta-mumick} and dynamical one \cite{scheuermann, shim, kalnis, chang}.
The former is based on a fixed amount of queries, whereas in the latter queries
for reservation are chosen based on appearance statistic and computational cost
of query execution. As a data source Data Warehouses or Relational Databases in
the hierarchical form are used. Special queries for data manipulation are
considered. In this paper we use split dimension generation for the data
required for current analysis. Hence, we are interested in standard SQL queries
to the Database and IDRs. The resulting representation can then be processed by
special multidimensional data queries.

In the \cite{Keller96} paper, they analyze the cache content,
and predicates in their work correspond to IDRs in our approach. Cached data
usage problem is resolved in terms of predicate deduction. In our paper, we address this problem by means of
computation of cached query results domains that correspond to the IDRs. It allows
us to define IR records that can be used to form a new representation and new SQL
queries that will let us load missing data from Database server. The following
example demonstrates suggested approach.


{\bf Example 1.} The task is to receive data representation as in the table
\ref{def_T_1}. Dimension attributes are in bold, measure attributes are italic,
else are attribute values.

\begin{table}[h!]
\caption{\label{def_T_1} Curriculum.}
\begin{center}
\begin{tabular}{|c|c||c|c|c|c|}
\cline{2-6}
\multicolumn{1}{c|}{} & \multicolumn{1}{|r||}{\bf Semester} & \multicolumn{2}{|c|}{2} & \multicolumn{2}{|c|}{3}  \\
\cline{2-6}
\multicolumn{1}{c|}{} & {\bf Discipline} & Physics & Chemistry & Physics & Chemistry \\
\hline \cline{2-6}
{\bf Department} & {\bf Class} & {\it Hours} & {\it Hours} & {\it Hours} & {\it Hours} \\
\hline
 & C-321 & & 36 & 18 & 9 \\
\cline{2-6}
D. of Chemistry & C-322 & & 48 & 18 & 9 \\
\cline{2-6}
 & C-331 & & 24 & 18 & 9 \\
\hline
 & P-321 & 48 & & 24 & \\
\cline{2-6}
D. of Physics & P-322 & 48 & & 24 & \\
\cline{2-6}
 & P-331 & 64 & & 24 & \\
\hline
\end{tabular}
\end{center}
\end{table}
Assume the precomputed relations on the user PC:  $P_1$,
$P_2$, $P_3$ with the following schemes: $P_1$=(Department number, Department), without any data constraints; $P_2$=(Discipline number, Discipline), without data constraints; $P_3$=(Department number, Discipline number, Semester, Class, Hours), with the following constraint:
$F_3 =(\mbox{Semester} \leq 2) \wedge (\mbox{Department} = \mbox{D. of Chemistry})$.
Assume the user has made the following logical constraint on the data to be represented in the table: $F^{\ast} = (\mbox{Semester} \leq 3)$. Comparing these formulas domains, it's easy to see that $P_1$ and $P_2$ have enough data to form table \ref{def_T_1}, whereas $P_3$ needs to be expanded with data. After defining $F^{\ast}$ and $F_3$ domains we can form a new formula: $F=((\mbox{Semester} = 3) \wedge (\mbox{Deparmtent=D. of Chemistry}))
\vee ((\mbox{Semester} \leq 3) \wedge (\mbox{Deparmtent} \neq \mbox{D. of Chemistry}))$. Using $P_3$ schema and $F$ formula, we can now make an SQL command to be executed on the server. As a result, we transfer less data over the network.

\section{Logical constraint definition and properties}

To simplify domain computations we will consider logical formulas in Disjunctive Normal Form (DNF).
In general case formula $F$ has the following form:

\begin{equation}
F = K_1 \vee K_2 \vee \dots \vee K_m ,
\label{def_F_1}
\end{equation}
\begin{equation}
K_i = T_1 \& T_2 \dots \& T_n, i = 1, \dots, m ,
\label{def_F_2}
\end{equation}
here $T_j, j = 1, \dots, n$ - predicates, where expanded attribute names are specified. $R_i.A_j$ means attribute $A_j$ in relation $R_i$.
Those predicates can be:
\begin{itemize}
  \item	comparison operation $ \n{Expr}_1 \theta \n{Expr}_2$, $\theta$ –
  comparison operator $(\theta \in \{=, \neq, >, <, \leq, \geq\})$, $\n{Expr}_i$
  – type conformant expressions, defined in a space of expanded attribute names and constants;
  \item operation $\n{Expr}_1 \n{[NOT]} \n{BETWEEN} \n{Expr}_2 \n{AND}
  \n{Expr}_3$ (symbols inside square brackets $[*]$ aren't mandatory for the predicate);
  \item operation $\n{Expr} \n{[NOT]} \n{IN} S$, here $S$ – values list or subquery, having a set of attribute $R_i.A_j$ values as a result;
  \item operation $\n{Str}_1 \n{[NOT]} \n{LIKE} \n{Str}_2$, here $\n{Str}_i$ –
  strings;
  \item operation $\n{Expr} \theta \n{ALL/ANY} S$.
\end{itemize}

\begin{note}
We assume logical formulas have no trivial conditions on attributes, for example
, $Expr_1 = Expr_1$ and those reduced to such form. In general, we assume that
$R_i.A_l$ domain is not fully contained in $T_j (\dots , R_i . A_l ,\dots )$
predicate truth space. Such conditions can be removed from the formula without a
change in the truth space (we will define it later).
\label{trivial}
\end{note}

\begin{mydef}
Space of attributes contained in a formula show the dimension of the formula and
is denoted as $\<F> $.

\begin{equation}
\<F> = \{R_1^F.A_1^F, \dots, R_k^F.A_k^F\}
\label{def_F_3}
\end{equation}
\end{mydef}


Listed variants of operations don't use every single SQL capability. For example
, we don't use $\n{EXISTS}$ predicate, because it has no expanded attribute
names in it. $\n{NULL}$ predicate is used for another purpose in our paper.

During logical formula domain calculation if we have some attribute having
$NULL$ value on a row $t$, we then get the $\n{UNKNOWN}$ value for the whole
formula, because SQL-query results follow the Three-valued logic. It leads us to
ambiguous interpretations of result of both usual users and experienced
programmers. To solve this problem we suggest the following constraint: every
attribute in $F^{\ast}$ is supplied with a property ``Use of undefined value''
with two mutually exclusive values: ``Yes'' or ``No''. The reasoning behind
this property is the following: if it is assigned ``Yes'', then we leave rows
with $\n{NULL}$ value for further consideration. Otherwise, having ``No'' in
that property guarantees us removing all such rows.

Let's write expression (\ref{def_F_1}) for $F$ in the following form: $ F (\dots
, T_j , \dots )$, here $T_j$ - predicates of expression (\ref{def_F_2}).
Then after the modification it will be the following: 
$F( \dots, T'_j , \dots )\wedge_{i,j}(R_i .A_j \neq  \n{NULL})$, here
$\wedge_{i,j}(R_i .A_j \neq  \n{NULL})$ - conjunction of all $F$ attributes,for
which $\n{NULL}$ is not allowed, and $T'_j = (T_j \vee_{i,j}(R_i .A_j =
\n{NULL}))$, here  $\vee_{i,j}(R_i .A_j = \n{NULL})$ - disjunction of all $F$
attributes, for which $\n{NULL}$ is allowed. Outer brackets for $T'_j$ predicate
define operation priority. We can see that this logical formula can only have
$\n{TRUE}$ and $\n{FALSE}$ values when considering it in Three-valued logic. We
also can note that if rows don't have undefined values, the initial formula $F$
will be equivalent to the transformed one. There can be some cases when the
meaning of $P$ IR changes with such a transformation. For example, assume $F =
R_1 .A_2 > 3 \vee R_3 .A_4 < 4$. Let row $t$ has $R_1 .A_2$ value set to
$\n{NULL}$, with a property ``Use of undefined value'' set to ``Yes'', and $R_3
.A_4$ value set to 5, so $R_3 .A_4 < 4$ predicate is $\n{FALSE}$. Then $F$
transformation on $t$ row will be $\n{TRUE}$, what is not so obvious indeed.

Hereafter we will suppose all the $F$ formulas to be transformed.

We consider $\mathcal{A} =$ $\{(a_1, \dots, a_n)
\mid a_i \in Dom(A_i), i=1,\dots,n\}$ set, here $Dom(A_i)$ - is a domain of
$A_i$. Cartesian product $Dom(A_1)\times Dom(A_2)\times \dots \times Dom(A_n)$
is an $n$-dimensional space of all values for all Database attributes. Data
constraints bound this space to some set of points that represents available
Database states. In the Note \ref{trivial} we consider the whole space, not only
the available states.

\begin{mydef}
Truth space of logical formula $F$, defined by (\ref{def_F_1}),
(\ref{def_F_2}), (\ref{def_F_3}), is a set $M (F) = \{a \in \mathcal{A} \mid
F(a) = \n {TRUE}\}$.
\end{mydef}

\begin{note}
$F$ formula dimension can be smaller than $\mathcal{A}$ dimension. In this case
attributes that aren't presented in formula can take any values in their
domains.
\end{note}

\begin{note}
The complexity of $\n{Expr}$, $S$ and $\n{Str}$ predicates is defined by software ability to compute Truth spaces of formulas, as it is shown in example 1.
\end{note}

We now see how Truth space operations can be handled with respect to the corresponding formulas. $M (F)$ for a given formula $F$, written in DNF, is nothing but a union of Truth spaces of distinct conjunctive clauses. Truth space of a single conjunctive clause is an intersection of Truth spaces of it's predicates.

\begin{mydef}
Given a logical formula $F$, defined by (\ref{def_F_1}), (\ref{def_F_2}),
(\ref{def_F_3}), the projection of $F$ on the $X$ attribute set is a logical formula $F[X], \<F[X]> = X$, that has all its predicates with $R_i^F.A_i^F \notin X$ replaced with trivial predicate $\n{TRUE}$.
\label{projection}
\end{mydef}

\begin{statement}[Inclusion property]
$\forall X \subseteq \<F> \quad M(F) \subseteq M(F[X])$
\label{proj_property_of_inclusion}
\end{statement}
\begin{proof}
If $X = \<F> $, then $F = F[X]$ and $M(F) = M(F[X])$. Assume $X \subset \<F> $.
Consider arbitrary point $a \in M(F)$, i.e. $F(a)=\n{TRUE}$. $F$ to $F[X]$
transformation is made by replacing $T_j$ predicates that contain $A_j \in \<F> 
$ attribute, such that $A_j \notin X$ with $\n{TRUE}$ value. We get
$F([X])=\n{TRUE}$ according to (\ref{def_F_1}) and (\ref{def_F_2}). Therefore,
$a \in M(F[X])$.
\end{proof}

This property of logical formulas is used while building new data representation
from given IDRs.

\section{Intermediate Data Representation properties research}

We denote saved IDRs as $P$=$\{ P_1$, $P_2$, $\dots$, $P_m \}$, here $P_v =
\pi_{X_v}(\sigma_{F_v} (R^v_1 \Join R^v_2 \Join \dots \Join R^v_{s(v)} ))$,
$s(v)$ – amount of relations in Database that were used while building $P_v$ IDR,
$\pi_{X_v}$ - projection on $X_v$ attributes, $\sigma_{F_v}$ - selection with
$F_v$ logical formula. We need to get the following representation out of $P$
IDRs:
$$P^{\ast} = \pi_{X^{\ast}}(\sigma_{F^{\ast}} (R^{\ast}_1 \Join R^{\ast}_2\Join
\dots \Join R^{\ast}_l )$$

Lets study the problem of building data representation $P^{\ast}$ using existing
$P_v$ IDRs.

\begin{theorem}
$P^{\ast} \subseteq \pi_{X^{\ast}} (\sigma_{F^{\ast}[X_v]} (P_{v}))$
, if:
\\a) $X^{\ast} \subseteq X_{v}$
\\b) 
$ \{R^{v}_{1}, \ldots, R^{v}_{s(v)}\}
\subseteq
\{R^{\ast}_{1}, \ldots, R^{\ast}_{l}\} $
\\c) $M (F^{\ast}) \subseteq M (F_{v}) $.
\label{th_base}
\end{theorem} 
\begin{proof}
Let $t^{\ast}$ be any tuple in $P^{\ast}$. We need to show that
$t^{\ast} \in \pi_{X^{\ast}} (P_{v})$. From condition  $t^{\ast} \in P^{\ast}$,
there is the tuple $t' \in R^{\ast}_1 \Join R^{\ast}_2\Join\ldots
\Join R^{\ast}_l$ and $t^{\ast} = t'[X^{\ast}]$. There may be several such tuples. We will choose the one that satisfies $F^{\ast}(t')=TRUE$.
This tuple exists indeed, because otherwise $t^{\ast}$ would not be in $P^{\ast}$. Thus, there are tuples
$t^{\ast}_i \in R^{\ast}_i$:

\begin{equation}
t^{\ast}_i = t'[\langle R^{\ast}_i\rangle], i = 1,\dots,l,
\label{eval_eq_1}
\end{equation}
\def \intersecij {\langle R^{\ast}_i \rangle \cap \langle R^{\ast}_j \rangle}
and for any pair $i$ and $j$, such that  $\intersecij \neq \emptyset$, equality is held:
\begin{equation}
t^{\ast}_i[\intersecij] = t'_i[\intersecij], i,j = 1,\dots,l,
\label{eval_eq_2}
\end{equation}

Whereas conditions (\ref{eval_eq_1}) and (\ref{eval_eq_2}) are held for the
whole set of relations $\{R^{\ast}_{1}, \ldots, R^{\ast}_{l}\}$, they are
true for any its subset also, including $\{R^{v}_{1}, \ldots, R^{v}_{s(v)}\}$.
Hence, after joining tuples $t^{\ast}_i$ from relations $R^{v}_j$ we get a tuple
$t''$, such that $t'' = t'[\langle {\Join}_{i=1}^{s(v)} R^v_i \rangle]$.
Equality $t''[X^{\ast}] = t'[X^{\ast}] = t^{\ast}$ follows from a) and $X_v
\subseteq \langle {\Join}_{i=1}^{s(v)} R^v_i \rangle$.

Condition  $F^{\ast}(t') = \n{TRUE}$ and statement 
\ref{proj_property_of_inclusion} imply the truth of the projection $F^{\ast}
[X^{\ast}]$ on a tuple $t'$, and consequently on $t''$, as the formula is defined at common attributes of these tuples. Furthermore, according to condition 3, we have $F_v (t') = \n{TRUE} \Rightarrow F_v (t'') = \n{TRUE}$.
It means $t^{\ast} = t''[X^{\ast}] \in \pi_{X^{\ast}} (P_{v})$. The theorem is
proved.
\end{proof}


The conditions given in theorem guarantee the data to build $P^{\ast}$ to be contained in $P_v$ IDR. However, there can be excess rows that make $F^{\ast}$ take $\n{TRUE}$ value. They appear because there are some relations in $R^{\ast}_1 \Join R^{\ast}_2\Join \dots \Join R^{\ast}_l $ that aren't presented in $R^v_1 \Join R^v_2 \Join \dots \Join R^v_{s(v)} $ and will be deleted if we join those missing relations. Using Truth spaces of logical formulas we can query DBMS to get a minimal required data set to define excess rows.

In the next theorem we describe the case of coinciding relation sets.

\begin{theorem}
$P^{\ast} = \pi_{X^{\ast}} (P_{v})$, if:
\\a) $X^{\ast} \subseteq X_{v}$
\\b) $\{R^{v}_{1}, \ldots, R^{v}_{s(v)}\} = \{R^{\ast}_{1}, \ldots,
R^{\ast}_{l}\}$
\\c) $M (F^{\ast}) = M (F_{v}) $.
\label{th_base_eq}
\end{theorem}
\begin{proof}
Conditions of the theorem are a special case of theorem \ref{th_base}, so
inclusion $P^{\ast} \subseteq \pi_{X^{\ast}} (P_{v})$ is considered  to be
proven. It is necessary to show that $\pi_{X^{\ast}} (P_{v}) \subseteq
P^{\ast}$. Assume tuple $t_v \in \pi_{X^{\ast}} (P_{v})$. Let's show that $t_v
\in P^{\ast}$. From definition of tuple $t_v$, it follows that there is tuple
$t' \in R^v_1 \Join R^v_2\Join\ldots \Join R^v_{s(v)}$ and $t_v = t'[X^{\ast}]$,
$F_v(t')=\n{TRUE}$. According to the commutativity of natural join, $R^v_1 \Join
R^v_2\Join\ldots \Join R^v_{s(v)} = R^{\ast}_1 \Join R^{\ast}_2\Join\ldots \Join
R^{\ast}_l$. Therefore, $t' \in R^{\ast}_1 \Join R^{\ast}_2\Join\ldots \Join
R^{\ast}_l$. From condition c) we have $F^{\ast} (t')=\n{TRUE}$. Hence, $t_v \in
P^{\ast}$. The theorem has been proven.
\end{proof}


It is possible to use several IRs to build resulting representation. At first,
let's consider the following simple property of natural join.

\begin{statement}
Assume $\Re_1 = R_1 \Join \dots \Join R_k$ - natural join of some $k$ relations. Also assume $\Re_2 = R_1 \Join \dots \Join R_k \Join R_{k+1} \Join \dots \Join R_{n}$. Then $\Re_2 [\langle \Join_{i=1}^{k} R_i \rangle] \subseteq \Re_1$
\label{join_property}
\end{statement}
\begin{proof}
It is so indeed, because joining additional relations to $\Re_1$ can only remove some rows that were presented there. After the projection, we will get the same relation $\Re_1$ if no rows were deleted or some subset otherwise. The order of joins doesn't matter thanks to commutativity property of this operation.
\end{proof}

\author{Мосин Сергей, Зыкин Сергей}
\def \bigcupn {\bigcup\limits_{v=1}^{n}}
\begin{theorem}
$P^{\ast} \subseteq \pi_{X^{\ast}} ( \sigma_{F^{\ast}[X]} (P_1 \bowtie \dots \bowtie P_n))$, где $X = \bigcupn X_{v}$ если:
\\а) $X^{\ast} \subseteq X$
\\б)
$ \bigcupn \{R^{v}_{1}, \ldots, R^{v}_{s(v)}\} = \{R'_{1}, \ldots, R'_{s'}\}
\subseteq
\{R^{\ast}_{1}, \ldots, R^{\ast}_{l}\} $
\\в) $M(F^{\ast}) \subseteq M(F_{v}), v = 1,\dots,n $.

\label{th_mult}
\end{theorem}
\begin{proof}
Аналогично, выберем некий кортеж $t^{\ast} \in P^{\ast}$ и покажем, 
что $t^{\ast} \in \pi_{X^{\ast}} ( \sigma_{F^{\ast}[X]} (\bowtie_{v=1}^{n} P_{v}))$. Строим кортежи
$t', t'' = t'[\langle {\bowtie}_{i=1}^{s'} R'_i \rangle]$ таким же образом,
как в Теореме \ref{th_base}. Заметим, что $\bigcupn X_{v} \subseteq 
{\bowtie}_{i=1}^{s'} \langle R'_i \rangle$, следовательно, получим
$t''[X^{\ast}] = t'[X^{\ast}] = t^{\ast}$.
Дальнейшие рассуждения также повторяются, распространяясь на все промежуточные представления. Получаем $F^{\ast} [X] (t'') = \n{TRUE}$ и $F_v (t'') = \n{TRUE}, v = 1,\dots,n $.
По утверждению \ref{join_property} $t''[\langle \bowtie_{i=1}^{s(v)} R^v_i \rangle]
\in 
R^v_1 \bowtie \dots \bowtie R^v_{s(v)}, v = 1, \dots, n
\Rightarrow   
t''[X_v] \in P_v, v = 1, \dots, n
\Rightarrow
t''[\bigcupn X_{v}] \in P_1 \bowtie \dots \bowtie P_n$.
А это значит, что $t^{\ast} = t''[X^{\ast}] \in 
\pi_{X^{\ast}} ( \sigma_{F^{\ast}[X]} (P_1 \bowtie \dots \bowtie P_n))$.
Теорема доказана.
\end{proof}


Just like in previous case, there are some additional conditions that allow us to get precise data presentation from IRs.

%\author{Mosin Sergey, Zykin Sergey}
\begin{theorem}
$P^{\ast} = \pi_{X^{\ast}} (P_1 \Join \dots \Join 
P_n)$, where $X = \bigcupn X_{v}$ if:
\\a) $X^{\ast} \subseteq X$, $X_v \supseteq \<\Join_{i=1}^{s(v)} R^v_i> \cap (\bigcup\limits_{\substack{w=1\\ w \neq v}}^{n} \<\Join_{i=1}^{s(w)} R^w_i> ), v = 1,\dots,n$
\\b)
$ \bigcupn \{R^{v}_{1}, \ldots, R^{v}_{s(v)}\} = \{R'_{1}, \ldots, R'_{s'}\}
= \{R^{\ast}_{1}, \ldots, R^{\ast}_{l}\} $
\\c) $M(F^{\ast}) = M(F_{1} \& \dots \& F_{n})$.
\label{th_mult_eq}
\end{theorem} 
\begin{proof}
Conditions of the theorem are a special case of theorem \ref{th_base}, so
inclusion  $P^{\ast} \subseteq \pi_{X^{\ast}} (\Join_{v=1}^{n} P_{v})$ is
considered to be proved. It is necessary to show, that $\pi_{X^{\ast}}
(\Join_{v=1}^{n} P_{v}) \subseteq P^{\ast}$. Assume tuple $t \in \pi_{X^{\ast}}
(\Join_{v=1}^{n} P_{v})$. Let's show, that $t \in P^{\ast}$. Using the property
of the projection operation and condition a) we have $\pi_{X^{\ast}}(\pi_{X_1}
(R^1_1 \Join \dots \Join R^1_{s(1)}) \Join \dots \Join \pi_{X_n}(R^n_1 \Join
\dots \Join R^n_{s(n)})) = \pi_{X^{\ast}} (R'_1 \Join \dots \Join R'_{s'})$.
Thus, according to the definition of tuple $t$, there is a tuple  $t' \in R'_1
\Join \ldots \Join R'_{s'}$ and $t = t'[X^{\ast}]$, $F_v(t')=\n{TRUE}, v =
1,\dots,n$. According to commutativity of natural join, $R'_1 \Join \ldots \Join
R'_s = R^{\ast}_1 \Join \ldots \Join R^{\ast}_l$, hence, $t' \in R^{\ast}_1 \Join
\ldots \Join R^{\ast}_l$. From condition c) we have $F^{\ast} (t')=\n{TRUE}$. Therefore $t \in P^{\ast}$. The theorem is proved.
\end{proof} 


The conditions listed above aren't exceptional in any case, because the set of dimensions that form hypercube changes rarely, so we can create new dimensions (data representations) from IRs with much less DB interaction.

\section{Conclusion}
The prime result of our paper is the justification of IR usage conditions for
new hypercube creation. Those IRs can be stored on an analyst's PC and
drastically shorten the time of getting the required data.

The common problem of systems that don't use any DBMS is the data refreshing. In
MOLAP approach, the technical support staff is responsible for periodical
refreshing of data. The same method can be used to refresh $P_v$
representations. The access to the DMBS's DB modification log and refreshing
only modified data decreases the refresh time period. However, let's notice
again that those dimensions don't change often.

We assume the usage of GPU to implement intermediate filtration and union of
different data representations. The research in this field shows that GPU
operations are much faster than those on CPU for small data. There is a decrease
in performance with data scale. The reason is obvious: big amount of data leads
to a much more frequent data transfer between different memory types and
increases the overall time elapsed. But the constant GPU memory growth and the
fact that data amount required for multidimensional data dimensions building is
comparatively small make the usage of graphical processors quite a good choice
for these purposes.
\begin{thebibliography}{9}
\bibitem{codd}
  \textit{Codd E.F., Codd S.B., Salley C.T.}
  Providing OLAP (On-line Analytical Processing) to User-Analysts: An IT
  Mandate. – Sunnyvale(CA): Codd \& Date Inc.,
  1993.
  – 31 p.
\bibitem{lecht}
  \textit{Lechtenborger J.}, \textit{Vossen G.}
  Multidimensional normal forms for data warehouse design
  // Inf. Syst.
  – 2003.
  – Vol. 28, N 5. – P. 415–434. 
\bibitem{lehner}
  \textit{Lehner W., Albrecht J., Wedekind H.}
  Normal forms for multidimensional databases
  // Proc. of the Tenth Intern. Conf. on Scientific and Statistical Database
  Management.
  – Capri, 1998.
  – P. 63–72.
\bibitem{mazon}
  \textit{Mazon J.-N., Trujillo J., Lechtenborger J.}
  Reconciling requirement-driven data warehouses with data sources via
  multidimensional normal forms
  // Data Knowledge Engineering.
  – 2007.
  – Vol. 63, N 3. – P. 725–751.
\bibitem{vassi}
  \textit{Vassiliadis P., Sellis T.}
  A survey of logical models for OLAP databases
  // SIGMOD Rec.
  – 1999.
  – Vol. 28, N 4. – P. 64–69.
\bibitem{peder}
  \textit{Pedersen T.B., Jensen C.S., Dyreson C.E.}
  A foundation for capturing and querying complex multidimensional data
  // Inf. Syst.
  – 2001.
  – Vol. 26, N 5. – P. 383–423.
\bibitem{progressive}
  \textit{Progressive ranking of range aggregates}
  / H.-G. Li, et al.
  // Data \& Knowledge Engineering.
  – 2007.
  – Vol. 63, N 1. – P. 4–25. 
\bibitem{giorg}
  \textit{Giorgini P., Rizzi S., Garzetti M.}
  Goal-oriented requirement analysis for data warehouse design
  // Proc. of the 8th ACM international Workshop on Data Warehousing and OLAP: DOLAP '05.
  – Bremen, 2005.
  – P. 47–56.

%ours  
\bibitem{zyk_pol}
  \textit{Sergey Zykin, Andrey Poluyanov}
  Multidimensional data building using Intermediate Representations.
  "Administration problems".
  2013.
  \textnumero 5. P. 54-59.

%stream
\bibitem{Olston03}
  \textit{Olston C., Jiang J., Widom J.}
  Adaptive filters for continuous queries over distributed data streams
  // Proc. of the 2003 ACM SIGMOD Intern. Conf. on Management of Data (SIGMOD '03).
  – San Diego, 2003.
  – P. 563–574.
\bibitem{Denny05}
  \textit{Denny M., Franklin M.J.}
  Predicate result range caching for continuous queries
  // Proc. of the 2005 ACM SIGMOD Intern. Conf. on Management of Data (SIGMOD '05).
  – N.-Y., 2005.
  – P. 646–657.

%logic
\bibitem{Afrati06}
  \textit{Afrati F., Li C., Mitra P.}
  Rewriting queries using views in the presence of arithmetic comparisons
  // Theoretical Computer Science.
  – 2006.
 – Vol. 368, N 1–2. – P. 88–123.

%static approach
\bibitem{baralis}
  \textit{Baralis, E., Paraboschi, S., Teniente, E.} Materialized view selection in a multidimensional database.
  // Proc. of the 23rd International Conference on Very Large Data Bases,
  Athens, Greece.
  - 1997
  - P. 318–329
\bibitem{gupta}
  \textit{Gupta, H.}
  Selection of views to materialize in a data warehouse.
  // Proceedings of the International Conference on Database Theory, Delphi.
  - Greece, 1997.
  - P. 98–112.
\bibitem{gupta-mumick}
  \textit{Gupta, H., Mumick, I.S.}
  Selection of views to materialize under a maintenance cost constraint.
  // Proceedings of the International Conference on Database Theory.
  - Israel, 1999.
  - P. 453–470.

%dynamic
\bibitem{scheuermann}
  \textit{Scheuermann, P., Shim, J., Vingralek R.}
  WATCHMAN: A data warehouse intelligent cache manager.
  // Proceedings of the 22nd International Conference on Very Large Data Bases.
  - Bombay, India, 1996.
  - P. 51–62.
\bibitem{shim}
  \textit{Shim, J., Scheuermann, P., Vingralek R.}
  Dynamic caching of query results for decision support systems.
  // Proceedings of the 11th International Conference on Scientific and
  Statistical Database Management.
  - Cleveland, OH, 1999.
  - P. 254–263.
\bibitem{kalnis}
  \textit{Kalnis, P., Papadias, D.}
  Proxy-server architectures for OLAP.
  // Proceedings of the 2001 ACM SIGMOD International Conference on Management
  of Data.
  - Santa Barbara, CA, 2001.
  - P. 367–378.
\bibitem{chang}
  \textit{Chang-Sup Park, Myoung Ho Kim b, Yoon-Joon Lee}
  Usability-based caching of query results in OLAP systems.
  // The Journal of Systems and Software
  - 2003
  - Vol. 68, - P. 103–119

%domain
\bibitem{Keller96}
  \textit{Keller M., Basu J.}
  A predicate-based caching scheme for client-server database architectures
  // VLDB Journal. – 1996. N 5. – P. 35–47.
  
  
\end{thebibliography}

\end{document}