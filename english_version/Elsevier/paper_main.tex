%% 
%% Copyright 2007, 2008, 2009 Elsevier Ltd
%% 
%% This file is part of the 'Elsarticle Bundle'.
%% ---------------------------------------------
%% 
%% It may be distributed under the conditions of the LaTeX Project Public
%% License, either version 1.2 of this license or (at your option) any
%% later version.  The latest version of this license is in
%%    http://www.latex-project.org/lppl.txt
%% and version 1.2 or later is part of all distributions of LaTeX
%% version 1999/12/01 or later.
%% 
%% The list of all files belonging to the 'Elsarticle Bundle' is
%% given in the file `manifest.txt'.
%% 

%% Template article for Elsevier's document class `elsarticle'
%% with numbered style bibliographic references
%% SP 2008/03/01

\documentclass[preprint,12pt]{elsarticle}

%% Use the option review to obtain double line spacing
%% \documentclass[authoryear,preprint,review,12pt]{elsarticle}

%% Use the options 1p,twocolumn; 3p; 3p,twocolumn; 5p; or 5p,twocolumn
%% for a journal layout:
%% \documentclass[final,1p,times]{elsarticle}
%% \documentclass[final,1p,times,twocolumn]{elsarticle}
%% \documentclass[final,3p,times]{elsarticle}
%% \documentclass[final,3p,times,twocolumn]{elsarticle}
%% \documentclass[final,5p,times]{elsarticle}
%% \documentclass[final,5p,times,twocolumn]{elsarticle}

%% For including figures, graphicx.sty has been loaded in
%% elsarticle.cls. If you prefer to use the old commands
%% please give \usepackage{epsfig}

%% The amssymb package provides various useful mathematical symbols
\author{Mosin Sergey, Zykin Sergey}
\usepackage{amssymb,amsmath}
\usepackage{amsthm}
%\usepackage{amsfonts}
%\usepackage{algorithm}
%\usepackage{xcolor}
%\usepackage{textcomp}
\newtheorem{note}{Note}
\newtheorem{theorem}{Theorem}
\newtheorem{definition}{Definition}
\newtheorem{statement}{Statement}
\def \eval #1#2{\left.#1\right\vert_{#2}}
\def \<#1> {\langle #1 \rangle}
\def \n #1{\mathit{#1}}
%% The amsthm package provides extended theorem environments
%% \usepackage{amsthm}

%% The lineno packages adds line numbers. Start line numbering with
%% \begin{linenumbers}, end it with \end{linenumbers}. Or switch it on
%% for the whole article with \linenumbers.
%% \usepackage{lineno}

\journal{Theoretical Computer Science}

\begin{document}

\begin{frontmatter}

%% Title, authors and addresses

%% use the tnoteref command within \title for footnotes;
%% use the tnotetext command for theassociated footnote;
%% use the fnref command within \author or \address for footnotes;
%% use the fntext command for theassociated footnote;
%% use the corref command within \author for corresponding author footnotes;
%% use the cortext command for theassociated footnote;
%% use the ead command for the email address,
%% and the form \ead[url] for the home page:
%% \title{Title\tnoteref{label1}}
%% \tnotetext[label1]{}
%% \author{Name\corref{cor1}\fnref{label2}}
%% \ead{email address}
%% \ead[url]{home page}
%% \fntext[label2]{}
%% \cortext[cor1]{}
%% \address{Address\fnref{label3}}
%% \fntext[label3]{}

\title{Truth space method for caching database queries}

%% use optional labels to link authors explicitly to addresses:
%% \author[label1,label2]{}
%% \address[label1]{}
%% \address[label2]{}

\author{S.~V. Mosin, S.~V. Zykin}

\address{Sobolev Institute of Mathematics of the Siberian Branch
of the Russian Academy of Sciences}

\begin{abstract}
We propose a new method of client-side data caching for relational databases
with a central server and distant clients. Data is loaded into client cache
based on queries executed on the central database at the server. These
queries have a special form - "universal relational query". The majority of
search queries to database can be expressed in such form. Besides, this form
allows us to analyze query correctness by checking lossless join property. A
subsequent query may be executed in client's local cache if we can determine
that the query result is entirely contained in the cache. For this we compare
truth spaces of the logical restrictions in new user's query and results of
the queries execution in cache. This method may be used to define lacking data
in cache and execute the query on server for this data only. Problem of data
actualizations in cache is not discussed in this paper. However, it can be
solved by cataloging of queries on server and their serving by triggers in
background mode.
\end{abstract}

\begin{keyword}
%% keywords here, in the form: keyword  keyword

%% PACS codes here, in the form: \PACS code \sep code

%% MSC codes here, in the form: \MSC code \sep code
%% or \MSC[2008] code \sep code (2000 is the default)
Relational databases \sep Caching \sep Truth space
\end{keyword}

\end{frontmatter}

%% \linenumbers

%% main text
\section{Introduction}
In this paper we discuss the use of data views that are
cached on the client's computer. We assume client-server environment with
server based Relational Database (RDB). The cache stores query results in order
to provide maximum usage of saved data in subsequent queries.

The purpose of this paper is to study the problem of building and using data
views on a client's computer. This problem is similar to query optimization
because it aims to decrease the data transfer from a Database server. Cached
data is actively used in Database Management Systems (DBMS), but mostly it is
only repeating use of data written in cache, without any prior data analysis
aimed to define any partial or combined use. The DBMS can only avoid requesting
blocks of data from external devices, while serving the query, if they are
present in the cache. So, only block numbers are analyzed, not their information.

This paper is based on the results, obtained in \cite{zyk_pol}. We have removed
a constraint that limits the choice of attributes for Intermediate Data View
(IDV) and also made some generalizations on multiple IDVs case.
The second section provides an overview of publications on the subject of this
paper. We have considered only those closest to the problem at hand. In the end
of this section we present an example to explain our approach for solving the
problem.

In the third section we provide a formalization of the problem. We also present
the approach for removing the uncertainties in database queries and present
auxiliary properties and definitions that are used later.

The main results are presented in Section 4. We research the possibility of
using cached results of previous queries when performing a new one. The results
are presented in the form of theorems.

% Head 1
\section{Existing solutions overview and comparison}
One of the main research problems of queries optimization to database is the
construction of an optimal query plan. Queries are transformed without content
analysis of the database content and cache. In other cases, this information is
taken into account for calculation statistical estimations to improve physical
access to data. These problems are the subject of many studies, but they are
outside of our approach.

A lot of publications are devoted to the problem of cache content management.
For example, the heuristic algorithms update the cache with regards to user data
\cite{chang}, storage of important user queries in a cache
\cite{baralis}, storage of the cached data on multiple servers \cite{kalnis}.
This paper solves the problem of the best use of a cache for execution of the
user's queries. Cache content management and the best use of the cache are two
complementary problem.

The most similar to our work is paper \cite{Afrati06}. The authors analyze
conjunctive queries on data domains with predicates in the form of arithmetical
comparisons, and present query computation algorithms using IDVs. In our paper,
the special case of universal relational query is considered. It is a query on
the Database relations, not particular domains. Although we had similar aims,
the results obtained differ because of stated factors. In particular, there is
no need to create any algorithms of data selection in our work, as they are
replaced by Relational Algebra.

In papers \cite{Keller96}, \cite{shim} similar problem is discussed. In our paper we
make correspondence between cache contents and predicates. Cached data usage
problem is resolved in terms of truth spaces. We compute the truth spaces of
query results in the cache. It allows us to define records in IDV that can be
used to form a new view and new SQL queries that will let us load missing data
from Database server. The following example demonstrates suggested approach.

{\bf Example 1.} Let's assume the following database schema fragment, which
represents the University Study plan:\\
$R_1 =\mbox{\it Students}\ (\mbox{\bf Stud\_ID}, \mbox{Stud\_Name},
\mbox{Group})$\\
$R_2 =\mbox{\it Schedule}\ (\mbox{\bf Group}, \mbox{\bf Room\_ID},
\mbox{Course})$\\
$R_3 =\mbox{\it Progress}\ (\mbox{\bf Stud\_ID}, \mbox{\bf Course},
\mbox{Score})$,\\
relation names are italic, Primary Key attributes are in bold. Assume that on 
he user's computer the following queries are cached:

Query 1: List of students studying physics, whose ID is bigger than 210:
$$P_1 = \pi_{X_1}(\sigma_{F_1} (R_1 \Join R_2)),$$
where $\pi_{X_1}$ -- projection operation over the set of attributes
$X_1 = \{\text{Stud\_ID},\\
\text{Stud\_Name}, \text{Group}, \text{Course}\}$,
$\sigma$ -- selecting operation,
$F_1$ -- logical formula: $(\text{Stud\_ID} > 210\ \&\ \text{Course} =
\text{Physics})$, $\Join$ --natural join.

Query 2: Examination sheets of the group M10:
$$P_2 = \pi_{X_2}(\sigma_{F_2} (R_1 \Join R_3)),$$
where
$X_2 = \{\text{Stud\_ID}, \text{Group}, \text{Course}, \text{Score}\}$, $F_2$
-- logical condition:\\
$(\text{Group}=\text{M10})$.

Let us assume that the user has requested information formalized with the
following query:
$$P^{\ast} = \pi_{X^{\ast}}(\sigma_{F^{\ast}} (R_1 \Join R_2 \Join R_3 )),$$
where $X^{\ast}= \{\text{Stud\_ID}, \text{Group}, \text{Score}\}$, $F^{\ast}$
-- logical condition: $(\text{Stud\_ID} > 300\ \&\ \text{Group} =
\text{M10}\ \&\ \text{Course} = \text{Physics})$.

Using the calculation and comparison of the truth domains $P_1$, $P_2$ and
$P^{\ast}$, we obtain that the query can be executed in the cache:
$P^{\ast} =\pi_{X^{\ast}}(\sigma_{F_3} (P_1 \Join P_2 ))$, where $F_3$
-- logical condition: $(\text{Stud\_ID} > 300)$. Requesting the server in this
case is not required.


\section{Logical constraint definition and properties}
To simplify domain computations we will consider logical formulas in
Disjunctive Normal Form (DNF). In general case formula $F$ has the following
form:

\begin{equation}
F = K_1 \vee K_2 \vee \dots \vee K_m ,
\label{def_F_1}
\end{equation}
\begin{equation}
K_i = T_1 \&\ T_2 \dots \&\ T_n, i = 1, \dots, m ,
\label{def_F_2}
\end{equation}
here $T_j, j = 1, \dots, n$ - predicates, where expanded attribute names are
specified. $R_i.A_j$ means attribute $A_j$ in relation $R_i$.
Those predicates can be:
\begin{itemize}
\item comparison operation $ \n{Expr}_1\ \theta\ \n{Expr}_2\ $, $\theta$
-- comparison operator $(\theta \in \{=, \neq, >, <, \leq, \geq\})$,
$\n{Expr}_i$ -- type conformant expressions, defined in a space of expanded
attribute names and constants;
\item operation $\n{Expr}_1\ \n{[NOT]}\ \n{BETWEEN}\ \n{Expr}_2\ \n{AND}\
\n{Expr}_3$
(symbols inside square brackets $[*]$ are optional);
\item operation $\n{Expr}\ \n{[NOT]}\ \n{IN}\ S$, here $S$ - values list or
subquery, having a set of attribute $R_i.A_j$ values as a result;
\item operation $\n{Str}_1\ \n{[NOT]}\ \n{LIKE}\ \n{Str}_2$, here $\n{Str}_i$
-- strings;
\item operation $\n{Expr}\ \theta\ \n{ALL/ANY}\ S$.
\end{itemize}

\begin{note}
We assume logical formulas have no trivial conditions on attributes, for
example, $Expr_1 = Expr_1$ and those reduced to such form. In general, we assume
that $R_i.A_l$ domain is not fully contained in
$T_j (\dots , R_i . A_l ,\dots )$ predicate truth space. Such conditions can be
removed from the formula without a change in the truth space (we will define it
later).
\label{trivial}
\end{note}

\begin{definition}
Space of attributes contained in a formula show the dimension of the formula and
is denoted as $\<F> $.

\begin{equation}
\<F> = \{R_1^F.A_1^F, \dots, R_k^F.A_k^F\} .
\label{def_F_3}
\end{equation}
\end{definition}

Listed variants of operations don't use every single SQL capability. For example,
we don't use the $\n{EXISTS}$ predicate, because it has no expanded attribute
names in it. The $\n{NULL}$ predicate is used for another purpose in our paper.

During logical formula domain calculation, if we have some attribute having 
$NULL$ value on a tuple $t$, we then get $\n{UNKNOWN}$ value for the whole
formula, because SQL-query results follow the Three-valued logic. It leads us to
ambiguous interpretations of results of both usual users and experienced
programmers. To solve this problem we suggest the following constraint: every
attribute in $F^{\ast}$ is supplied with a property ``Use of undefined value''
with two mutually exclusive values: ``Yes'' or ``No''. The reasoning behind this
property is the following: if it is assigned ``Yes'', then we leave tuples with
$\n{NULL}$ value for further consideration. Otherwise, having ``No'' in that
property guarantees us removing all such tuples.

Let's write expression (\ref{def_F_1}) for $F$ in the following form: $ F (\dots
, T_j , \dots )$, here $T_j$ -- predicates of expression (\ref{def_F_2}). After
the modification it will be the following: $F( \dots, T'_j , \dots )\wedge_{i,j}
(R_i .A_j \neq \n{NULL})$, here $\wedge_{i,j}(R_i .A_j \neq \n{NULL})$
-- conjunction of all $F$ attributes,for which $\n{NULL}$ is not allowed, and
$T'_j = (T_j \vee_{i,j}(R_i .A_j = \n{NULL}))$, here $\vee_{i,j}(R_i .A_j =
\n{NULL})$ -- disjunction of all $F$ attributes, for which $\n{NULL}$ is allowed.
Outer brackets for $T'_j$ predicate define operation priority. We can see that
this logical formula can only have $\n{TRUE}$ and $\n{FALSE}$ values when
considering it in Three-valued logic. We can also note that if tuples don't have
undefined values, the initial formula $F$ will be equivalent to the transformed
one, so semantics of view $P$ almost undistorted. For the disclosure of the term
"almost" we consider example. Assume $F = R_1.A_2 > 3 \vee R_3.A_4 < 4$. Let
tuple $t$ for $R_1.A_2$ has value $\n{NULL}$, with a property ``Use of undefined
value'' equal to ``Yes'', and $R_3.A_4$ value equal to 5, so value of
$R_3.A_4 < 4$ is $\n{FALSE}$. Then transform of formula $F$ on tuple $t$ will be
equal $\n{TRUE}$, which is not obvious.

Hereafter we will assume all the $F$ formulas to be transformed.

We consider set $\mathcal{A} =$ $\{(a_1, \dots, a_n) \mid a_i \in Dom(A_i),
i=1,\dots,n\}$, here $Dom(A_i)$ -- is a domain of $A_i$. Cartesian product
$Dom(A_1)\times Dom(A_2)\times \dots \times Dom(A_n)$ is an $n$-dimensional
space of all values for all Database attributes. Data constraints bound this
space to some set of points that represents a set of available Database
states.

In our example with University Study plan, the $\mathcal{A}$ can be the
following:

\begin{multline*}
\mathcal{A} = Dom(\text{Stud\_ID})\times Dom(\text{Stud\_Name}) \times
Dom(\text{Group}) \times\\
Dom(\text{Room\_ID}) \times Dom(\text{Course}) \times Dom(\text{Score}) =
\{205, 315, 461\} \times \{\text{Rachel}\\
\text{Davis, Noam Angrist}, \text{Cameron McCord}\} \times\{\text{M10, M11}\}
\times \{100, 101, 102,\\
103\} \times \{\text{Physics, Chemistry}\} \times \{70, 80, 90\}
\end{multline*}

Obviously, without constraints this set represents more states than DB can have.
For example, each student has his/her own ID, whereas there are 3 corresponding
IDs for each student in $\mathcal{A}$: $\{205, 315, 461\}\times \{\text{Rachel
Davis},\\
\text{Noam Angrist, Cameron McCord}\}$.

\begin{note}
Dimension of formula $F$ can be smaller than dimension of $\mathcal{A}$. In this
case, we consider the equivalent form of the formula that has all other
attributes taking any values in their domains.
\end{note}

Looking back to our Example 1, we see that the formula in Query 1 contains only
two attributes: $F_1 = (\text{Stud\_ID} > 210\ \&\ \text{Course} =
\text{Physics})$, in other words $\<F> = 2$. The equivalent representation that
we will use is the following:

\begin{multline*}
F_1 = (\text{Stud\_ID} > 210\ \&\ \text{Course} = \text{Physics}\ \&\
\text{Stud\_Name}\ \textit{IN}\ \{\text{Rachel Davis},\\
\text{Noam Angrist, Cameron McCord}\}\ \&\ \text{Group}\ \textit{IN}\ \{\text{M10, M11}\}\ \&\
\text{Room\_ID}\ \textit{IN}\\ \{100, 101, 102, 103\}\ \&\ \text{Score}\ \textit{IN}\ \{70, 80, 90\})
\end{multline*}

The motivation behind this transformation is to allow us comparing truth spaces
of formulas defined on different attributes.

\begin{definition}
Truth space of logical formula $F$, defined by (\ref{def_F_1}),
(\ref{def_F_2}), (\ref{def_F_3}), is a set $M (F) = \{a \in \mathcal{A} \mid
F(a) = \n {TRUE}\}$.
\end{definition}

So, the truth space of our formula $F_1$ will be the following subset of
$\mathcal{A}$: $M (F) = $ \{(315, Rachel Davis, M10, 100, Physics, 70), (315,
Noam Angrist, M10, 100, Physics, 70), \dots (461, Cameron McCord, M10, 100,
Physics, 90)\}

$M (F)$ for a given formula $F$, written in DNF, is nothing but a union of Truth
spaces of distinct conjunctive clauses. Truth space of a single conjunctive
clause is an intersection of Truth spaces of it's predicates.

\begin{note}
The complexity of $\n{Expr}$, $S$ and $\n{Str}$ predicates is defined by the
software's ability to compute Truth spaces of formulas, as it is shown in
example 1.
\end{note}

\begin{definition}
Given a logical formula $F$, defined by (\ref{def_F_1}), (\ref{def_F_2}),
(\ref{def_F_3}), the projection of $F$ on the $X$ attribute set is a logical
formula $F[X], \<F[X]> = X$, that has all its predicates with $R_i^F.A_i^F
\notin X$ replaced with trivial predicate $\n{TRUE}$.
\label{projection}
\end{definition}

So, the projection of $F_1$ on the attribute set $X = \{\text{Stud\_ID}\}$ will be
$F_1[\text{Stud\_ID}] = (\text{Stud\_ID} > 210\ \&\ \text{TRUE}) = (\text{Stud\_ID} >
210)$ 

\begin{statement}[Inclusion property]
$\forall X \subseteq \<F> \quad M(F) \subseteq M(F[X])$
\label{proj_property_of_inclusion}
\end{statement}
\begin{proof}
If $X = \<F> $, then $F = F[X]$ and $M(F) = M(F[X])$. Assume $X \subset \<F> $.
Consider arbitrary point $a \in M(F)$, i.e. $F(a)=\n{TRUE}$. $F$ to $F[X]$
transformation is made by replacing $T_j$ predicates that contain $A_j \in
\<F> $ attribute, such that $A_j \notin X$ with $\n{TRUE}$ value. We get
$F([X])=\n{TRUE}$ according to (\ref{def_F_1}) and (\ref{def_F_2}). Therefore,
$a \in M(F[X])$.
\end{proof}

This property of logical formulas is used while building new data view from given
IDVs.

\section{Intermediate Data View properties research}

We denote saved IDVs as $P$=$\{ P_1$, $P_2$, $\dots$, $P_m \}$, here $P_v$ is an
IDV, $P_v = \pi_{X_v}(\sigma_{F_v} (R^v_1 \Join R^v_2 \Join \dots \Join
R^v_{s(v)} ))$, $s(v)$ -- amount of relations in Database that were used while
building $P_v$,
$\pi_{X_v}$ -- projection on $X_v$ attributes, $\sigma_{F_v}$ -- selection
with $F_v$ logical formula. Eventually we need to get the following view:
$$P^{\ast} = \pi_{X^{\ast}}(\sigma_{F^{\ast}}
(R^{\ast}_1 \Join R^{\ast}_2\Join\dots \Join R^{\ast}_l ))$$
Let’s study the problem of building data view $P^{\ast}$ using existing $P_v$ IDVs.

\begin{theorem}
$P^{\ast} \subseteq \pi_{X^{\ast}} (\sigma_{F^{\ast}[X_v]} (P_{v}))$
, if:
\\a) $X^{\ast} \subseteq X_{v}$
\\b) 
$ \{R^{v}_{1}, \ldots, R^{v}_{s(v)}\}
\subseteq
\{R^{\ast}_{1}, \ldots, R^{\ast}_{l}\} $
\\c) $M (F^{\ast}) \subseteq M (F_{v}) $.
\label{th_base}
\end{theorem} 
\begin{proof}
Let $t^{\ast}$ be any tuple in $P^{\ast}$. We need to show that
$t^{\ast} \in \pi_{X^{\ast}} (P_{v})$. From condition  $t^{\ast} \in P^{\ast}$,
there is the tuple $t' \in R^{\ast}_1 \Join R^{\ast}_2\Join\ldots
\Join R^{\ast}_l$ and $t^{\ast} = t'[X^{\ast}]$. There may be several such tuples. We will choose the one that satisfies $F^{\ast}(t')=TRUE$.
This tuple exists indeed, because otherwise $t^{\ast}$ would not be in $P^{\ast}$. Thus, there are tuples
$t^{\ast}_i \in R^{\ast}_i$:

\begin{equation}
t^{\ast}_i = t'[\langle R^{\ast}_i\rangle], i = 1,\dots,l,
\label{eval_eq_1}
\end{equation}
\def \intersecij {\langle R^{\ast}_i \rangle \cap \langle R^{\ast}_j \rangle}
and for any pair $i$ and $j$, such that  $\intersecij \neq \emptyset$, equality is held:
\begin{equation}
t^{\ast}_i[\intersecij] = t'_i[\intersecij], i,j = 1,\dots,l,
\label{eval_eq_2}
\end{equation}

Whereas conditions (\ref{eval_eq_1}) and (\ref{eval_eq_2}) are held for the
whole set of relations $\{R^{\ast}_{1}, \ldots, R^{\ast}_{l}\}$, they are
true for any its subset also, including $\{R^{v}_{1}, \ldots, R^{v}_{s(v)}\}$.
Hence, after joining tuples $t^{\ast}_i$ from relations $R^{v}_j$ we get a tuple
$t''$, such that $t'' = t'[\langle {\Join}_{i=1}^{s(v)} R^v_i \rangle]$.
Equality $t''[X^{\ast}] = t'[X^{\ast}] = t^{\ast}$ follows from a) and $X_v
\subseteq \langle {\Join}_{i=1}^{s(v)} R^v_i \rangle$.

Condition  $F^{\ast}(t') = \n{TRUE}$ and statement 
\ref{proj_property_of_inclusion} imply the truth of the projection $F^{\ast}
[X^{\ast}]$ on a tuple $t'$, and consequently on $t''$, as the formula is defined at common attributes of these tuples. Furthermore, according to condition 3, we have $F_v (t') = \n{TRUE} \Rightarrow F_v (t'') = \n{TRUE}$.
It means $t^{\ast} = t''[X^{\ast}] \in \pi_{X^{\ast}} (P_{v})$. The theorem is
proved.
\end{proof}


The conditions given in the aforementioned theorem guarantee the data to build
$P^{\ast}$ to be contained in $P_v$ IDV. However, there can be excess tuples that
make $F^{\ast}$
take $\n{TRUE}$ value. They appear because there are some relations in
$R^{\ast}_1 \Join R^{\ast}_2\Join \dots \Join R^{\ast}_l $ that aren't
presented in $R^v_1 \Join R^v_2 \Join \dots \Join R^v_{s(v)} $ and will be
deleted if we join those missing relations. Using Truth spaces of logical
formulas we can query DBMS to get a minimal required data set to define excess
tuples.

In the next theorem we describe the case of coinciding relation sets.

\begin{theorem}
$P^{\ast} = \pi_{X^{\ast}} (P_{v})$, if:
\\a) $X^{\ast} \subseteq X_{v}$
\\b) $\{R^{v}_{1}, \ldots, R^{v}_{s(v)}\} = \{R^{\ast}_{1}, \ldots,
R^{\ast}_{l}\}$
\\c) $M (F^{\ast}) = M (F_{v}) $.
\label{th_base_eq}
\end{theorem}
\begin{proof}
Conditions of the theorem are a special case of theorem \ref{th_base}, so
inclusion $P^{\ast} \subseteq \pi_{X^{\ast}} (P_{v})$ is considered  to be
proven. It is necessary to show that $\pi_{X^{\ast}} (P_{v}) \subseteq
P^{\ast}$. Assume tuple $t_v \in \pi_{X^{\ast}} (P_{v})$. Let's show that $t_v
\in P^{\ast}$. From definition of tuple $t_v$, it follows that there is tuple
$t' \in R^v_1 \Join R^v_2\Join\ldots \Join R^v_{s(v)}$ and $t_v = t'[X^{\ast}]$,
$F_v(t')=\n{TRUE}$. According to the commutativity of natural join, $R^v_1 \Join
R^v_2\Join\ldots \Join R^v_{s(v)} = R^{\ast}_1 \Join R^{\ast}_2\Join\ldots \Join
R^{\ast}_l$. Therefore, $t' \in R^{\ast}_1 \Join R^{\ast}_2\Join\ldots \Join
R^{\ast}_l$. From condition c) we have $F^{\ast} (t')=\n{TRUE}$. Hence, $t_v \in
P^{\ast}$. The theorem has been proven.
\end{proof}


It is possible to use several IDVs to build a resulting representation. First,
let's consider the following simple property of natural join.

\begin{statement}
Assume $\Re_1 = R_1 \Join \dots \Join R_k$ - natural join of some $k$ relations.
Also assume $\Re_2 = R_1 \Join \dots \Join R_k \Join R_{k+1} \Join \dots \Join
R_{n}$. Then $\Re_2 [\langle \Join_{i=1}^{k} R_i \rangle] \subseteq \Re_1$
\label{join_property}
\end{statement}
\begin{proof}
It is so indeed, because joining additional relations to $\Re_1$ can only
remove some tuples that were presented there. After the projection, we will get
the same relation $\Re_1$ if no tuples were deleted or some subset otherwise. The
order of joins doesn't matter because of commutativity of this operation.
\end{proof}

\author{Мосин Сергей, Зыкин Сергей}
\def \bigcupn {\bigcup\limits_{v=1}^{n}}
\begin{theorem}
$P^{\ast} \subseteq \pi_{X^{\ast}} ( \sigma_{F^{\ast}[X]} (P_1 \bowtie \dots \bowtie P_n))$, где $X = \bigcupn X_{v}$ если:
\\а) $X^{\ast} \subseteq X$
\\б)
$ \bigcupn \{R^{v}_{1}, \ldots, R^{v}_{s(v)}\} = \{R'_{1}, \ldots, R'_{s'}\}
\subseteq
\{R^{\ast}_{1}, \ldots, R^{\ast}_{l}\} $
\\в) $M(F^{\ast}) \subseteq M(F_{v}), v = 1,\dots,n $.

\label{th_mult}
\end{theorem}
\begin{proof}
Аналогично, выберем некий кортеж $t^{\ast} \in P^{\ast}$ и покажем, 
что $t^{\ast} \in \pi_{X^{\ast}} ( \sigma_{F^{\ast}[X]} (\bowtie_{v=1}^{n} P_{v}))$. Строим кортежи
$t', t'' = t'[\langle {\bowtie}_{i=1}^{s'} R'_i \rangle]$ таким же образом,
как в Теореме \ref{th_base}. Заметим, что $\bigcupn X_{v} \subseteq 
{\bowtie}_{i=1}^{s'} \langle R'_i \rangle$, следовательно, получим
$t''[X^{\ast}] = t'[X^{\ast}] = t^{\ast}$.
Дальнейшие рассуждения также повторяются, распространяясь на все промежуточные представления. Получаем $F^{\ast} [X] (t'') = \n{TRUE}$ и $F_v (t'') = \n{TRUE}, v = 1,\dots,n $.
По утверждению \ref{join_property} $t''[\langle \bowtie_{i=1}^{s(v)} R^v_i \rangle]
\in 
R^v_1 \bowtie \dots \bowtie R^v_{s(v)}, v = 1, \dots, n
\Rightarrow   
t''[X_v] \in P_v, v = 1, \dots, n
\Rightarrow
t''[\bigcupn X_{v}] \in P_1 \bowtie \dots \bowtie P_n$.
А это значит, что $t^{\ast} = t''[X^{\ast}] \in 
\pi_{X^{\ast}} ( \sigma_{F^{\ast}[X]} (P_1 \bowtie \dots \bowtie P_n))$.
Теорема доказана.
\end{proof}


Just like in the previous case, there are some additional conditions that allow
us to get precise data presentation from IDVs.

%\author{Mosin Sergey, Zykin Sergey}
\begin{theorem}
$P^{\ast} = \pi_{X^{\ast}} (P_1 \Join \dots \Join 
P_n)$, where $X = \bigcupn X_{v}$ if:
\\a) $X^{\ast} \subseteq X$, $X_v \supseteq \<\Join_{i=1}^{s(v)} R^v_i> \cap (\bigcup\limits_{\substack{w=1\\ w \neq v}}^{n} \<\Join_{i=1}^{s(w)} R^w_i> ), v = 1,\dots,n$
\\b)
$ \bigcupn \{R^{v}_{1}, \ldots, R^{v}_{s(v)}\} = \{R'_{1}, \ldots, R'_{s'}\}
= \{R^{\ast}_{1}, \ldots, R^{\ast}_{l}\} $
\\c) $M(F^{\ast}) = M(F_{1} \& \dots \& F_{n})$.
\label{th_mult_eq}
\end{theorem} 
\begin{proof}
Conditions of the theorem are a special case of theorem \ref{th_base}, so
inclusion  $P^{\ast} \subseteq \pi_{X^{\ast}} (\Join_{v=1}^{n} P_{v})$ is
considered to be proved. It is necessary to show, that $\pi_{X^{\ast}}
(\Join_{v=1}^{n} P_{v}) \subseteq P^{\ast}$. Assume tuple $t \in \pi_{X^{\ast}}
(\Join_{v=1}^{n} P_{v})$. Let's show, that $t \in P^{\ast}$. Using the property
of the projection operation and condition a) we have $\pi_{X^{\ast}}(\pi_{X_1}
(R^1_1 \Join \dots \Join R^1_{s(1)}) \Join \dots \Join \pi_{X_n}(R^n_1 \Join
\dots \Join R^n_{s(n)})) = \pi_{X^{\ast}} (R'_1 \Join \dots \Join R'_{s'})$.
Thus, according to the definition of tuple $t$, there is a tuple  $t' \in R'_1
\Join \ldots \Join R'_{s'}$ and $t = t'[X^{\ast}]$, $F_v(t')=\n{TRUE}, v =
1,\dots,n$. According to commutativity of natural join, $R'_1 \Join \ldots \Join
R'_s = R^{\ast}_1 \Join \ldots \Join R^{\ast}_l$, hence, $t' \in R^{\ast}_1 \Join
\ldots \Join R^{\ast}_l$. From condition c) we have $F^{\ast} (t')=\n{TRUE}$. Therefore $t \in P^{\ast}$. The theorem is proved.
\end{proof} 


The conditions listed above aren't exceptional in any case. Many applications
require repeat of data entry with slight modifications. For example,
applications that
work with multidimensional data. Hypercube dimensions do not change often, so
that we can create new hypercubes from IDVs without frequent interaction with
the database. 

\section{Conclusion}
The approach suggested in this paper is a theoretical base for a system
interacting with Relational Database. Such system will be capable of defining
cached data availability when executing consecutive queries. This solution is
fresh and wasn't used before. The conditions stated in the theorems don't
require hitting Database to be checked. Results obtained allow us to
analitycally define data missing in cache and query only this data.

The proposed solution will be used in dynamic building of multidimensional data.
Multidimensional data building systems that use redundant data often face the
problem of updating it. It is sometimes solved by the periodic updating of the
hypercube contents.
A similar method can be used to update the views $P_v$. To reduce the update
time it might be a good practice to use the change log on the database server
and update only those views that have original data changed. However, those
dimensions don't change often.

\section{Acknowledgements}
This work is supported by the Russian Foundation for Basic Research under
grant \#12-07-00066-a.

\bibliographystyle{elsarticle-num}
\bibliography{references}

\end{document}
\endinput
