%\author{Mosin Sergey, Zykin Sergey}
\begin{theorem}
$P^{\ast} = \pi_{X^{\ast}} (P_1 \Join \dots \Join 
P_n)$, where $X = \bigcupn X_{v}$ if:
\\a) $X^{\ast} \subseteq X$, $X_v \supseteq \<\Join_{i=1}^{s(v)} R^v_i> \cap (\bigcup\limits_{\substack{w=1\\ w \neq v}}^{n} \<\Join_{i=1}^{s(w)} R^w_i> ), v = 1,\dots,n$
\\b)
$ \bigcupn \{R^{v}_{1}, \ldots, R^{v}_{s(v)}\} = \{R'_{1}, \ldots, R'_{s'}\}
= \{R^{\ast}_{1}, \ldots, R^{\ast}_{l}\} $
\\c) $M(F^{\ast}) = M(F_{1} \& \dots \& F_{n})$.
\label{th_mult_eq}
\end{theorem} 
\begin{proof}
Conditions of the theorem are a special case of theorem \ref{th_base}, so
inclusion  $P^{\ast} \subseteq \pi_{X^{\ast}} (\Join_{v=1}^{n} P_{v})$ is
considered to be proved. It is necessary to show, that $\pi_{X^{\ast}}
(\Join_{v=1}^{n} P_{v}) \subseteq P^{\ast}$. Assume tuple $t \in \pi_{X^{\ast}}
(\Join_{v=1}^{n} P_{v})$. Let's show, that $t \in P^{\ast}$. Using the property
of the projection operation and condition a), we have $\pi_{X^{\ast}}(\pi_{X_1}
(R^1_1 \Join \dots \Join R^1_{s(1)}) \Join \dots \Join \pi_{X_n}(R^n_1 \Join
\dots \Join R^n_{s(n)})) = \pi_{X^{\ast}} (R'_1 \Join \dots \Join R'_{s'})$.
Thus, according to the definition of tuple $t$, there is a tuple  $t' \in R'_1
\Join \ldots \Join R'_{s'}$ and $t = t'[X^{\ast}]$, $F_v(t')=\n{TRUE}, v =
1,\dots,n$. According to the commutativity of natural join, $R'_1 \Join \ldots \Join
R'_s = R^{\ast}_1 \Join \ldots \Join R^{\ast}_l$, hence, $t' \in R^{\ast}_1 \Join
\ldots \Join R^{\ast}_l$. From condition c) we have $F^{\ast} (t')=\n{TRUE}$.
Therefore $t \in P^{\ast}$. The theorem has been proven.
\end{proof} 
