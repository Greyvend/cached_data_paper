\begin{theorem}
$P^{\ast} \subseteq \pi_{X^{\ast}} (\sigma_{F^{\ast}[X_v]} (P_{v}))$
, if:
\\a) $X^{\ast} \subseteq X_{v}$
\\b) 
$ \{R^{v}_{1}, \ldots, R^{v}_{s(v)}\}
\subseteq
\{R^{\ast}_{1}, \ldots, R^{\ast}_{l}\} $
\\c) $M (F^{\ast}) \subseteq M (F_{v}) $.
\label{th_base}
\end{theorem} 
\begin{proof}
Let $t^{\ast}$ be any tuple in $P^{\ast}$. We need to show that
$t^{\ast} \in \pi_{X^{\ast}} (P_{v})$. From condition  $t^{\ast} \in P^{\ast}$,
there is the tuple $t' \in R^{\ast}_1 \Join R^{\ast}_2\Join\ldots
\Join R^{\ast}_l$ and $t^{\ast} = t'[X^{\ast}]$. There may be several such tuples. We will choose the one that satisfies $F^{\ast}(t')=TRUE$.
This tuple exists indeed, because otherwise $t^{\ast}$ would not be in $P^{\ast}$. Thus, there are tuples
$t^{\ast}_i \in R^{\ast}_i$:

\begin{equation}
t^{\ast}_i = t'[\langle R^{\ast}_i\rangle], i = 1,\dots,l,
\label{eval_eq_1}
\end{equation}
\def \intersecij {\langle R^{\ast}_i \rangle \cap \langle R^{\ast}_j \rangle}
and for any pair $i$ and $j$, such that  $\intersecij \neq \emptyset$, equality is held:
\begin{equation}
t^{\ast}_i[\intersecij] = t'_i[\intersecij], i,j = 1,\dots,l,
\label{eval_eq_2}
\end{equation}

Whereas conditions (\ref{eval_eq_1}) and (\ref{eval_eq_2}) are held for the
whole set of relations $\{R^{\ast}_{1}, \ldots, R^{\ast}_{l}\}$, they are
true for any its subset also, including $\{R^{v}_{1}, \ldots, R^{v}_{s(v)}\}$.
Hence, after joining tuples $t^{\ast}_i$ from relations $R^{v}_j$ we get a tuple
$t''$, such that $t'' = t'[\langle {\Join}_{i=1}^{s(v)} R^v_i \rangle]$.
Equality $t''[X^{\ast}] = t'[X^{\ast}] = t^{\ast}$ follows from a) and $X_v
\subseteq \langle {\Join}_{i=1}^{s(v)} R^v_i \rangle$.

Condition  $F^{\ast}(t') = \n{TRUE}$ and statement 
\ref{proj_property_of_inclusion} imply the truth of the projection $F^{\ast}
[X^{\ast}]$ on a tuple $t'$, and consequently on $t''$, as the formula is defined at common attributes of these tuples. Furthermore, according to condition 3, we have $F_v (t') = \n{TRUE} \Rightarrow F_v (t'') = \n{TRUE}$.
It means $t^{\ast} = t''[X^{\ast}] \in \pi_{X^{\ast}} (P_{v})$. The theorem is
proved.
\end{proof}
