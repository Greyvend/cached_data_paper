%\author{Mosin Sergey, Zykin Sergey}
\def \bigcupn {\bigcup\limits_{v=1}^{n}}
\begin{theorem}
$P^{\ast} \subseteq \pi_{X^{\ast}} ( \sigma_{F^{\ast}[X]} (P_1 \Join \dots \Join P_n))$, where $X = \bigcupn X_{v}$ if:
\\a) $X^{\ast} \subseteq X$
\\b)
$ \bigcupn \{R^{v}_{1}, \ldots, R^{v}_{s(v)}\} = \{R'_{1}, \ldots, R'_{s'}\}
\subseteq
\{R^{\ast}_{1}, \ldots, R^{\ast}_{l}\} $
\\c) $M(F^{\ast}) \subseteq M(F_{v}), v = 1,\dots,n $.

\label{th_mult}
\end{theorem}
\begin{proof}
Again, we will choose an arbitrary tuple $t^{\ast} \in P^{\ast}$
and show that $t^{\ast} \in \pi_{X^{\ast}} ( \sigma_{F^{\ast}[X]}
(\Join_{v=1}^{n} P_{v}))$. By analogy to theorem \ref{th_base}, the tuples $t',
t'' = t'[\langle {\Join}_{i=1}^{s'} R'_i \rangle]$ are built.
Note that  $\bigcupn X_{v} \subseteq {\Join}_{i=1}^{s'} \langle R'_i \rangle$, hence
$t''[X^{\ast}] = t'[X^{\ast}] = t^{\ast}$.
The further reasonings also apply to all IRs.
We get $F^{\ast} [X] (t'') = \n{TRUE}$ and $F_v (t'') = \n{TRUE}, v = 1,\dots,n $.
Considering statement \ref{join_property}, we have $t''[\langle \Join_{i=1}^{s(v)} R^v_i \rangle]
\in 
R^v_1 \Join \dots \Join R^v_{s(v)}, v = 1, \dots, n
\Rightarrow   
t''[X_v] \in P_v, v = 1, \dots, n
\Rightarrow
t''[\bigcupn X_{v}] \in P_1 \Join \dots \Join P_n$.
It means that $t^{\ast} = t''[X^{\ast}] \in 
\pi_{X^{\ast}} ( \sigma_{F^{\ast}[X]} (P_1 \Join \dots \Join P_n))$.
The theorem has been proven.
\end{proof}
