
%% bare_jrnl_compsoc.tex
%% V1.3
%% 2007/01/11
%% by Michael Shell
%% See:
%% http://www.michaelshell.org/
%% for current contact information.
%%
%% This is a skeleton file demonstrating the use of IEEEtran.cls
%% (requires IEEEtran.cls version 1.7 or later) with an IEEE Computer
%% Society journal paper.
%%
%% Support sites:
%% http://www.michaelshell.org/tex/ieeetran/
%% http://www.ctan.org/tex-archive/macros/latex/contrib/IEEEtran/
%% and
%% http://www.ieee.org/

%%*************************************************************************
%% Legal Notice:
%% This code is offered as-is without any warranty either expressed or
%% implied; without even the implied warranty of MERCHANTABILITY or
%% FITNESS FOR A PARTICULAR PURPOSE! 
%% User assumes all risk.
%% In no event shall IEEE or any contributor to this code be liable for
%% any damages or losses, including, but not limited to, incidental,
%% consequential, or any other damages, resulting from the use or misuse
%% of any information contained here.
%%
%% All comments are the opinions of their respective authors and are not
%% necessarily endorsed by the IEEE.
%%
%% This work is distributed under the LaTeX Project Public License (LPPL)
%% ( http://www.latex-project.org/ ) version 1.3, and may be freely used,
%% distributed and modified. A copy of the LPPL, version 1.3, is included
%% in the base LaTeX documentation of all distributions of LaTeX released
%% 2003/12/01 or later.
%% Retain all contribution notices and credits.
%% ** Modified files should be clearly indicated as such, including  **
%% ** renaming them and changing author support contact information. **
%%
%% File list of work: IEEEtran.cls, IEEEtran_HOWTO.pdf, bare_adv.tex,
%%                    bare_conf.tex, bare_jrnl.tex, bare_jrnl_compsoc.tex
%%*************************************************************************

% *** Authors should verify (and, if needed, correct) their LaTeX system  ***
% *** with the testflow diagnostic prior to trusting their LaTeX platform ***
% *** with production work. IEEE's font choices can trigger bugs that do  ***
% *** not appear when using other class files.                            ***
% The testflow support page is at:
% http://www.michaelshell.org/tex/testflow/




% Note that the a4paper option is mainly intended so that authors in
% countries using A4 can easily print to A4 and see how their papers will
% look in print - the typesetting of the document will not typically be
% affected with changes in paper size (but the bottom and side margins will).
% Use the testflow package mentioned above to verify correct handling of
% both paper sizes by the user's LaTeX system.
%
% Also note that the "draftcls" or "draftclsnofoot", not "draft", option
% should be used if it is desired that the figures are to be displayed in
% draft mode.
%
% The Computer Society usually requires 10pt for submissions.
%
\documentclass[10pt,journal,cspaper,compsoc]{IEEEtran}
%
% If IEEEtran.cls has not been installed into the LaTeX system files,
% manually specify the path to it like:
% \documentclass[12pt,journal,compsoc]{../sty/IEEEtran}





% Some very useful LaTeX packages include:
% (uncomment the ones you want to load)


% *** MISC UTILITY PACKAGES ***
%
%\usepackage{ifpdf}
% Heiko Oberdiek's ifpdf.sty is very useful if you need conditional
% compilation based on whether the output is pdf or dvi.
% usage:
% \ifpdf
%   % pdf code
% \else
%   % dvi code
% \fi
% The latest version of ifpdf.sty can be obtained from:
% http://www.ctan.org/tex-archive/macros/latex/contrib/oberdiek/
% Also, note that IEEEtran.cls V1.7 and later provides a builtin
% \ifCLASSINFOpdf conditional that works the same way.
% When switching from latex to pdflatex and vice-versa, the compiler may
% have to be run twice to clear warning/error messages.






% *** CITATION PACKAGES ***
%
\ifCLASSOPTIONcompsoc
  % IEEE Computer Society needs nocompress option
  % requires cite.sty v4.0 or later (November 2003)
  % \usepackage[nocompress]{cite}
\else
  % normal IEEE
  % \usepackage{cite}
\fi
% cite.sty was written by Donald Arseneau
% V1.6 and later of IEEEtran pre-defines the format of the cite.sty package
% \cite{} output to follow that of IEEE. Loading the cite package will
% result in citation numbers being automatically sorted and properly
% "compressed/ranged". e.g., [1], [9], [2], [7], [5], [6] without using
% cite.sty will become [1], [2], [5]--[7], [9] using cite.sty. cite.sty's
% \cite will automatically add leading space, if needed. Use cite.sty's
% noadjust option (cite.sty V3.8 and later) if you want to turn this off.
% cite.sty is already installed on most LaTeX systems. Be sure and use
% version 4.0 (2003-05-27) and later if using hyperref.sty. cite.sty does
% not currently provide for hyperlinked citations.
% The latest version can be obtained at:
% http://www.ctan.org/tex-archive/macros/latex/contrib/cite/
% The documentation is contained in the cite.sty file itself.
%
% Note that some packages require special options to format as the Computer
% Society requires. In particular, Computer Society  papers do not use
% compressed citation ranges as is done in typical IEEE papers
% (e.g., [1]-[4]). Instead, they list every citation separately in order
% (e.g., [1], [2], [3], [4]). To get the latter we need to load the cite
% package with the nocompress option which is supported by cite.sty v4.0
% and later. Note also the use of a CLASSOPTION conditional provided by
% IEEEtran.cls V1.7 and later.





% *** GRAPHICS RELATED PACKAGES ***
%
\ifCLASSINFOpdf
  % \usepackage[pdftex]{graphicx}
  % declare the path(s) where your graphic files are
  % \graphicspath{{../pdf/}{../jpeg/}}
  % and their extensions so you won't have to specify these with
  % every instance of \includegraphics
  % \DeclareGraphicsExtensions{.pdf,.jpeg,.png}
\else
  % or other class option (dvipsone, dvipdf, if not using dvips). graphicx
  % will default to the driver specified in the system graphics.cfg if no
  % driver is specified.
  % \usepackage[dvips]{graphicx}
  % declare the path(s) where your graphic files are
  % \graphicspath{{../eps/}}
  % and their extensions so you won't have to specify these with
  % every instance of \includegraphics
  % \DeclareGraphicsExtensions{.eps}
\fi
% graphicx was written by David Carlisle and Sebastian Rahtz. It is
% required if you want graphics, photos, etc. graphicx.sty is already
% installed on most LaTeX systems. The latest version and documentation can
% be obtained at: 
% http://www.ctan.org/tex-archive/macros/latex/required/graphics/
% Another good source of documentation is "Using Imported Graphics in
% LaTeX2e" by Keith Reckdahl which can be found as epslatex.ps or
% epslatex.pdf at: http://www.ctan.org/tex-archive/info/
%
% latex, and pdflatex in dvi mode, support graphics in encapsulated
% postscript (.eps) format. pdflatex in pdf mode supports graphics
% in .pdf, .jpeg, .png and .mps (metapost) formats. Users should ensure
% that all non-photo figures use a vector format (.eps, .pdf, .mps) and
% not a bitmapped formats (.jpeg, .png). IEEE frowns on bitmapped formats
% which can result in "jaggedy"/blurry rendering of lines and letters as
% well as large increases in file sizes.
%
% You can find documentation about the pdfTeX application at:
% http://www.tug.org/applications/pdftex





% *** MATH PACKAGES ***
%
%\usepackage[cmex10]{amsmath}
% A popular package from the American Mathematical Society that provides
% many useful and powerful commands for dealing with mathematics. If using
% it, be sure to load this package with the cmex10 option to ensure that
% only type 1 fonts will utilized at all point sizes. Without this option,
% it is possible that some math symbols, particularly those within
% footnotes, will be rendered in bitmap form which will result in a
% document that can not be IEEE Xplore compliant!
%
% Also, note that the amsmath package sets \interdisplaylinepenalty to 10000
% thus preventing page breaks from occurring within multiline equations. Use:
%\interdisplaylinepenalty=2500
% after loading amsmath to restore such page breaks as IEEEtran.cls normally
% does. amsmath.sty is already installed on most LaTeX systems. The latest
% version and documentation can be obtained at:
% http://www.ctan.org/tex-archive/macros/latex/required/amslatex/math/





% *** SPECIALIZED LIST PACKAGES ***
%
%\usepackage{algorithmic}
% algorithmic.sty was written by Peter Williams and Rogerio Brito.
% This package provides an algorithmic environment fo describing algorithms.
% You can use the algorithmic environment in-text or within a figure
% environment to provide for a floating algorithm. Do NOT use the algorithm
% floating environment provided by algorithm.sty (by the same authors) or
% algorithm2e.sty (by Christophe Fiorio) as IEEE does not use dedicated
% algorithm float types and packages that provide these will not provide
% correct IEEE style captions. The latest version and documentation of
% algorithmic.sty can be obtained at:
% http://www.ctan.org/tex-archive/macros/latex/contrib/algorithms/
% There is also a support site at:
% http://algorithms.berlios.de/index.html
% Also of interest may be the (relatively newer and more customizable)
% algorithmicx.sty package by Szasz Janos:
% http://www.ctan.org/tex-archive/macros/latex/contrib/algorithmicx/




% *** ALIGNMENT PACKAGES ***
%
%\usepackage{array}
% Frank Mittelbach's and David Carlisle's array.sty patches and improves
% the standard LaTeX2e array and tabular environments to provide better
% appearance and additional user controls. As the default LaTeX2e table
% generation code is lacking to the point of almost being broken with
% respect to the quality of the end results, all users are strongly
% advised to use an enhanced (at the very least that provided by array.sty)
% set of table tools. array.sty is already installed on most systems. The
% latest version and documentation can be obtained at:
% http://www.ctan.org/tex-archive/macros/latex/required/tools/


%\usepackage{mdwmath}
%\usepackage{mdwtab}
% Also highly recommended is Mark Wooding's extremely powerful MDW tools,
% especially mdwmath.sty and mdwtab.sty which are used to format equations
% and tables, respectively. The MDWtools set is already installed on most
% LaTeX systems. The lastest version and documentation is available at:
% http://www.ctan.org/tex-archive/macros/latex/contrib/mdwtools/


% IEEEtran contains the IEEEeqnarray family of commands that can be used to
% generate multiline equations as well as matrices, tables, etc., of high
% quality.


%\usepackage{eqparbox}
% Also of notable interest is Scott Pakin's eqparbox package for creating
% (automatically sized) equal width boxes - aka "natural width parboxes".
% Available at:
% http://www.ctan.org/tex-archive/macros/latex/contrib/eqparbox/





% *** SUBFIGURE PACKAGES ***
%\ifCLASSOPTIONcompsoc
%\usepackage[tight,normalsize,sf,SF]{subfigure}
%\else
%\usepackage[tight,footnotesize]{subfigure}
%\fi
% subfigure.sty was written by Steven Douglas Cochran. This package makes it
% easy to put subfigures in your figures. e.g., "Figure 1a and 1b". For IEEE
% work, it is a good idea to load it with the tight package option to reduce
% the amount of white space around the subfigures. Computer Society papers
% use a larger font and \sffamily font for their captions, hence the
% additional options needed under compsoc mode. subfigure.sty is already
% installed on most LaTeX systems. The latest version and documentation can
% be obtained at:
% http://www.ctan.org/tex-archive/obsolete/macros/latex/contrib/subfigure/
% subfigure.sty has been superceeded by subfig.sty.


%\ifCLASSOPTIONcompsoc
%  \usepackage[caption=false]{caption}
%  \usepackage[font=normalsize,labelfont=sf,textfont=sf]{subfig}
%\else
%  \usepackage[caption=false]{caption}
%  \usepackage[font=footnotesize]{subfig}
%\fi
% subfig.sty, also written by Steven Douglas Cochran, is the modern
% replacement for subfigure.sty. However, subfig.sty requires and
% automatically loads Axel Sommerfeldt's caption.sty which will override
% IEEEtran.cls handling of captions and this will result in nonIEEE style
% figure/table captions. To prevent this problem, be sure and preload
% caption.sty with its "caption=false" package option. This is will preserve
% IEEEtran.cls handing of captions. Version 1.3 (2005/06/28) and later 
% (recommended due to many improvements over 1.2) of subfig.sty supports
% the caption=false option directly:
%\ifCLASSOPTIONcompsoc
%  \usepackage[caption=false,font=normalsize,labelfont=sf,textfont=sf]{subfig}
%\else
%  \usepackage[caption=false,font=footnotesize]{subfig}
%\fi
%
% The latest version and documentation can be obtained at:
% http://www.ctan.org/tex-archive/macros/latex/contrib/subfig/
% The latest version and documentation of caption.sty can be obtained at:
% http://www.ctan.org/tex-archive/macros/latex/contrib/caption/




% *** FLOAT PACKAGES ***
%
\usepackage{fixltx2e}
% fixltx2e, the successor to the earlier fix2col.sty, was written by
% Frank Mittelbach and David Carlisle. This package corrects a few problems
% in the LaTeX2e kernel, the most notable of which is that in current
% LaTeX2e releases, the ordering of single and double column floats is not
% guaranteed to be preserved. Thus, an unpatched LaTeX2e can allow a
% single column figure to be placed prior to an earlier double column
% figure. The latest version and documentation can be found at:
% http://www.ctan.org/tex-archive/macros/latex/base/



%\usepackage{stfloats}
% stfloats.sty was written by Sigitas Tolusis. This package gives LaTeX2e
% the ability to do double column floats at the bottom of the page as well
% as the top. (e.g., "\begin{figure*}[!b]" is not normally possible in
% LaTeX2e). It also provides a command:
%\fnbelowfloat
% to enable the placement of footnotes below bottom floats (the standard
% LaTeX2e kernel puts them above bottom floats). This is an invasive package
% which rewrites many portions of the LaTeX2e float routines. It may not work
% with other packages that modify the LaTeX2e float routines. The latest
% version and documentation can be obtained at:
% http://www.ctan.org/tex-archive/macros/latex/contrib/sttools/
% Documentation is contained in the stfloats.sty comments as well as in the
% presfull.pdf file. Do not use the stfloats baselinefloat ability as IEEE
% does not allow \baselineskip to stretch. Authors submitting work to the
% IEEE should note that IEEE rarely uses double column equations and
% that authors should try to avoid such use. Do not be tempted to use the
% cuted.sty or midfloat.sty packages (also by Sigitas Tolusis) as IEEE does
% not format its papers in such ways.




%\ifCLASSOPTIONcaptionsoff
%  \usepackage[nomarkers]{endfloat}
% \let\MYoriglatexcaption\caption
% \renewcommand{\caption}[2][\relax]{\MYoriglatexcaption[#2]{#2}}
%\fi
% endfloat.sty was written by James Darrell McCauley and Jeff Goldberg.
% This package may be useful when used in conjunction with IEEEtran.cls'
% captionsoff option. Some IEEE journals/societies require that submissions
% have lists of figures/tables at the end of the paper and that
% figures/tables without any captions are placed on a page by themselves at
% the end of the document. If needed, the draftcls IEEEtran class option or
% \CLASSINPUTbaselinestretch interface can be used to increase the line
% spacing as well. Be sure and use the nomarkers option of endfloat to
% prevent endfloat from "marking" where the figures would have been placed
% in the text. The two hack lines of code above are a slight modification of
% that suggested by in the endfloat docs (section 8.3.1) to ensure that
% the full captions always appear in the list of figures/tables - even if
% the user used the short optional argument of \caption[]{}.
% IEEE papers do not typically make use of \caption[]'s optional argument,
% so this should not be an issue. A similar trick can be used to disable
% captions of packages such as subfig.sty that lack options to turn off
% the subcaptions:
% For subfig.sty:
% \let\MYorigsubfloat\subfloat
% \renewcommand{\subfloat}[2][\relax]{\MYorigsubfloat[]{#2}}
% For subfigure.sty:
% \let\MYorigsubfigure\subfigure
% \renewcommand{\subfigure}[2][\relax]{\MYorigsubfigure[]{#2}}
% However, the above trick will not work if both optional arguments of
% the \subfloat/subfig command are used. Furthermore, there needs to be a
% description of each subfigure *somewhere* and endfloat does not add
% subfigure captions to its list of figures. Thus, the best approach is to
% avoid the use of subfigure captions (many IEEE journals avoid them anyway)
% and instead reference/explain all the subfigures within the main caption.
% The latest version of endfloat.sty and its documentation can obtained at:
% http://www.ctan.org/tex-archive/macros/latex/contrib/endfloat/
%
% The IEEEtran \ifCLASSOPTIONcaptionsoff conditional can also be used
% later in the document, say, to conditionally put the References on a 
% page by themselves.




% *** PDF, URL AND HYPERLINK PACKAGES ***
%
%\usepackage{url}
% url.sty was written by Donald Arseneau. It provides better support for
% handling and breaking URLs. url.sty is already installed on most LaTeX
% systems. The latest version can be obtained at:
% http://www.ctan.org/tex-archive/macros/latex/contrib/misc/
% Read the url.sty source comments for usage information. Basically,
% \url{my_url_here}.





% *** Do not adjust lengths that control margins, column widths, etc. ***
% *** Do not use packages that alter fonts (such as pslatex).         ***
% There should be no need to do such things with IEEEtran.cls V1.6 and later.
% (Unless specifically asked to do so by the journal or conference you plan
% to submit to, of course. )


% correct bad hyphenation here
\hyphenation{op-tical net-works semi-conduc-tor}

\author{Mosin Sergey, Zykin Sergey}
\usepackage{amssymb,amsmath}
\usepackage[utf8]{inputenc}
%\usepackage[russian]{babel}
\usepackage{amsthm}
\usepackage{amsfonts}
\usepackage{algorithmic}
\usepackage{algorithm}
\usepackage{xcolor}
\usepackage{textcomp}
\newtheorem{theorem}{Theorem}
\newtheorem{mydef}{Definition}
\newtheorem{statement}{Statement}
\newtheorem*{consequence}{Consequence}
\newtheorem{note}{Note}
\def \eval #1#2{\left.#1\right\vert_{#2}}
\def \<#1> {\langle #1 \rangle}
\def \n #1{\mathit{#1}}
\begin{document}
%
% paper title
% can use linebreaks \\ within to get better formatting as desired
\title{Truth space method for caching database queries}
%
%
% author names and IEEE memberships
% note positions of commas and nonbreaking spaces ( ~ ) LaTeX will not break
% a structure at a ~ so this keeps an author's name from being broken across
% two lines.
% use \thanks{} to gain access to the first footnote area
% a separate \thanks must be used for each paragraph as LaTeX2e's \thanks
% was not built to handle multiple paragraphs
%
%
%\IEEEcompsocitemizethanks is a special \thanks that produces the bulleted
% lists the Computer Society journals use for "first footnote" author
% affiliations. Use \IEEEcompsocthanksitem which works much like \item
% for each affiliation group. When not in compsoc mode,
% \IEEEcompsocitemizethanks becomes like \thanks and
% \IEEEcompsocthanksitem becomes a line break with idention. This
% facilitates dual compilation, although admittedly the differences in the
% desired content of \author between the different types of papers makes a
% one-size-fits-all approach a daunting prospect. For instance, compsoc 
% journal papers have the author affiliations above the "Manuscript
% received ..."  text while in non-compsoc journals this is reversed. Sigh.

\author{Sergey~V.~Mosin,
        Sergey~V.~Zykin % <-this % stops a space
\IEEEcompsocitemizethanks{\IEEEcompsocthanksitem S. V. Mosin, S. V. Zykin are
with the the Siberian Branch of the Russian Academy of Sciences, Omsk, Russia.
\protect\\
% note need leading \protect in front of \\ to get a newline within \thanks as
% \\ is fragile and will error, could use \hfil\break instead.
E-mail: svmosin@gmail.com, szykin@mail.ru}% <-this % stops a space
\thanks{}}

% note the % following the last \IEEEmembership and also \thanks - 
% these prevent an unwanted space from occurring between the last author name
% and the end of the author line. i.e., if you had this:
% 
% \author{....lastname \thanks{...} \thanks{...} }
%                     ^------------^------------^----Do not want these spaces!
%
% a space would be appended to the last name and could cause every name on that
% line to be shifted left slightly. This is one of those "LaTeX things". For
% instance, "\textbf{A} \textbf{B}" will typeset as "A B" not "AB". To get
% "AB" then you have to do: "\textbf{A}\textbf{B}"
% \thanks is no different in this regard, so shield the last } of each \thanks
% that ends a line with a % and do not let a space in before the next \thanks.
% Spaces after \IEEEmembership other than the last one are OK (and needed) as
% you are supposed to have spaces between the names. For what it is worth,
% this is a minor point as most people would not even notice if the said evil
% space somehow managed to creep in.



% The paper headers
\markboth{Journal of \LaTeX\ Class Files,~Vol.~6, No.~1, January~2007}%
{Shell \MakeLowercase{\textit{et al.}}: Truth space method for caching database
queries}
% The only time the second header will appear is for the odd numbered pages
% after the title page when using the twoside option.
% 
% *** Note that you probably will NOT want to include the author's ***
% *** name in the headers of peer review papers.                   ***
% You can use \ifCLASSOPTIONpeerreview for conditional compilation here if
% you desire.



% The publisher's ID mark at the bottom of the page is less important with
% Computer Society journal papers as those publications place the marks
% outside of the main text columns and, therefore, unlike regular IEEE
% journals, the available text space is not reduced by their presence.
% If you want to put a publisher's ID mark on the page you can do it like
% this:
%\IEEEpubid{0000--0000/00\$00.00~\copyright~2007 IEEE}
% or like this to get the Computer Society new two part style.
%\IEEEpubid{\makebox[\columnwidth]{\hfill 0000--0000/00/\$00.00~\copyright~2007 IEEE}%
%\hspace{\columnsep}\makebox[\columnwidth]{Published by the IEEE Computer Society\hfill}}
% Remember, if you use this you must call \IEEEpubidadjcol in the second
% column for its text to clear the IEEEpubid mark (Computer Society jorunal
% papers don't need this extra clearance.)




% for Computer Society papers, we must declare the abstract and index terms
% PRIOR to the title within the \IEEEcompsoctitleabstractindextext IEEEtran
% command as these need to go into the title area created by \maketitle.
\IEEEcompsoctitleabstractindextext{%
\begin{abstract}
%\boldmath
We propose a new method of client-side data caching for relational databases
with a central server and distant clients. Data is loaded into client cache
based on queries executed on the central database at the server. These
queries have a special form - "universal relational query". The majority of
search queries to database can be expressed in such form. Besides, this form
allows us to analyze query correctness by checking lossless join property. A
subsequent query may be executed in client's local cache if we can determine
that the query result is entirely contained in the cache. For this we compare
truth spaces of the logical restrictions in new user's query and results of
the queries execution in cache. This method may be used to define lacking data
in cache and execute the query on server for this data only. Problem of data
actualizations in cache is not discussed in this paper. However, it can be
solved by cataloging of queries on server and their serving by triggers in
background mode.
\end{abstract}
% IEEEtran.cls defaults to using nonbold math in the Abstract.
% This preserves the distinction between vectors and scalars. However,
% if the journal you are submitting to favors bold math in the abstract,
% then you can use LaTeX's standard command \boldmath at the very start
% of the abstract to achieve this. Many IEEE journals frown on math
% in the abstract anyway. In particular, the Computer Society does
% not want either math or citations to appear in the abstract.

% Note that keywords are not normally used for peer review papers.
\begin{keywords}
Relational databases, Caching, Truth space
\end{keywords}}


% make the title area
\maketitle


% To allow for easy dual compilation without having to reenter the
% abstract/keywords data, the \IEEEcompsoctitleabstractindextext text will
% not be used in maketitle, but will appear (i.e., to be "transported")
% here as \IEEEdisplaynotcompsoctitleabstractindextext when compsoc mode
% is not selected <OR> if conference mode is selected - because compsoc
% conference papers position the abstract like regular (non-compsoc)
% papers do!
\IEEEdisplaynotcompsoctitleabstractindextext
% \IEEEdisplaynotcompsoctitleabstractindextext has no effect when using
% compsoc under a non-conference mode.


% For peer review papers, you can put extra information on the cover
% page as needed:
% \ifCLASSOPTIONpeerreview
% \begin{center} \bfseries EDICS Category: 3-BBND \end{center}
% \fi
%
% For peerreview papers, this IEEEtran command inserts a page break and
% creates the second title. It will be ignored for other modes.
\IEEEpeerreviewmaketitle



\section{Introduction}
% Computer Society journal papers do something a tad strange with the very
% first section heading (almost always called "Introduction"). They place it
% ABOVE the main text! IEEEtran.cls currently does not do this for you.
% However, You can achieve this effect by making LaTeX jump through some
% hoops via something like:
%
%\ifCLASSOPTIONcompsoc
%  \noindent\raisebox{2\baselineskip}[0pt][0pt]%
%  {\parbox{\columnwidth}{\section{Introduction}\label{sec:introduction}%
%  \global\everypar=\everypar}}%
%  \vspace{-1\baselineskip}\vspace{-\parskip}\par
%\else
%  \section{Introduction}\label{sec:introduction}\par
%\fi
%
% Admittedly, this is a hack and may well be fragile, but seems to do the
% trick for me. Note the need to keep any \label that may be used right
% after \section in the above as the hack puts \section within a raised box.



% The very first letter is a 2 line initial drop letter followed
% by the rest of the first word in caps (small caps for compsoc).
% 
% form to use if the first word consists of a single letter:
% \IEEEPARstart{A}{demo} file is ....
% 
% form to use if you need the single drop letter followed by
% normal text (unknown if ever used by IEEE):
% \IEEEPARstart{A}{}demo file is ....
% 
% Some journals put the first two words in caps:
% \IEEEPARstart{T}{his demo} file is ....
% 
% Here we have the typical use of a "T" for an initial drop letter
% and "HIS" in caps to complete the first word.
\IEEEPARstart{I}n this paper we discuss the use of data views that are
cached on the client's computer. We assume client-server environment with
server based Relational Database (RDB). The cache stores query results in order
to provide maximum usage of saved data in subsequent queries.

The purpose of this paper is to study the problem of building and using data
views on a client's computer. This problem is similar to query optimization
because it aims to decrease the data transfer from a Database server. Cached
data is actively used in Database Management Systems (DBMS), but mostly it is
only repeating use of data written in cache, without any prior data analysis
aimed to define any partial or combined use. The DBMS can only avoid requesting
blocks of data from external devices, while serving the query, if they are
present in the cache. So, only block numbers are analyzed, not their information.

This paper is based on the results, obtained in \cite{zyk_pol}. We have removed
a constraint that limits the choice of attributes for Intermediate Data View
(IDV) and also made some generalizations on multiple IDVs case.

The second section provides an overview of publications on the subject of this
paper. We have considered only those closest to the problem at hand. In the end
of this section we present an example to explain our approach for solving the
problem.

In the third section we provide a formalization of the problem. We also present
the approach for removing the uncertainties in database queries and present
auxiliary properties and definitions that are used later.

The main results are presented in Section 4. We research the possibility of
using cached results of previous queries when performing a new one. The results
are presented in the form of theorems.

\section{Existing solutions overview and comparison}
One of the main research problems of queries optimization to database is the
construction of an optimal query plan. Queries are transformed without content
analysis of the database content and cache. In other cases, this information is
taken into account for calculation statistical estimations to improve physical
access to data. These problems are the subject of many studies, but they are
outside of our approach.

A lot of publications are devoted to the problem of cache content management.
For example, the heuristic algorithms update the cache with regards to user data
\cite{chang}, storage of important user queries in a cache
\cite{baralis}, storage of the cached data on multiple servers \cite{kalnis}.
This paper solves the problem of the best use of a cache for execution of the
user's queries. Cache content management and the best use of the cache are two
complementary problem.

The most similar to our work is paper \cite{Afrati06}. The authors analyze
conjunctive queries on data domains with predicates in the form of arithmetical
comparisons, and present query computation algorithms using IDVs. In our paper,
the special case of universal relational query is considered. It is a query on
the Database relations, not particular domains. Although we had similar aims,
the results obtained differ because of stated factors. In particular, there is
no need to create any algorithms of data selection in our work, as they are
replaced by Relational Algebra.

In papers \cite{Keller96, shim} similar problem is discussed. In our paper we
make correspondence between cache contents and predicates. Cached data usage
problem is resolved in terms of truth spaces. We compute the truth spaces of
query results in the cache. It allows us to define records in IDV that can be
used to form a new view and new SQL queries that will let us load missing data
from Database server. The following example demonstrates suggested approach.

{\bf Example 1.} Let's assume the following database schema fragment, which
represents the University Study plan:\\
$R_1 =\mbox{\it Students}\ (\mbox{\bf Stud\_ID}, \mbox{Stud\_Name},
\mbox{Group})$\\
$R_2 =\mbox{\it Schedule}\ (\mbox{\bf Group}, \mbox{\bf Room\_ID},
\mbox{Course})$\\
$R_3 =\mbox{\it Progress}\ (\mbox{\bf Stud\_ID}, \mbox{\bf Course},
\mbox{Score})$,\\
relation names are italic, Primary Key attributes are in bold. Assume that on 
he user's computer the following queries are cached:\\
Query 1: List of students studying physics, whose ID is bigger than 210:
$$P_1 = \pi_{X_1}(\sigma_{F_1} (R_1 \Join R_2)),$$
where $\pi_{X_1}$ -- projection operation over the set of attributes
$X_1 = \{\text{Stud\_ID}, \text{Stud\_Name}, \text{Group}, \text{Course}\}$,
$\sigma$ -- selecting operation,
$F_1$ -- logical formula: $(\text{Stud\_ID} > 210\ \&\ \text{Course} =
\text{Physics})$, $\Join$ --natural join.

Query 2: Examination sheets of the group M10:
$$P_2 = \pi_{X_2}(\sigma_{F_2} (R_1 \Join R_3)),$$
where
$X_2 = \{\text{Stud\_ID}, \text{Group}, \text{Course}, \text{Score}\}$, $F_2$
-- logical condition: $(\text{Group}=\text{M10})$.

Let us assume that the user has requested information formalized with the
following query:
$$P^{\ast} = \pi_{X^{\ast}}(\sigma_{F^{\ast}} (R_1 \Join R_2 \Join R_3 )),$$
where $X^{\ast}= \{\text{Stud\_ID}, \text{Group}, \text{Score}\}$, $F^{\ast}$
-- logical condition: $(\text{Stud\_ID} > 300\ \&\ \text{Group} =
\text{M10}\ \&\ \text{Course} = \text{Physics})$.

Using the calculation and comparison of the truth domains $P_1$, $P_2$ and
$P^{\ast}$, we obtain that the query can be executed in the cache:
$P^{\ast} =\pi_{X^{\ast}}(\sigma_{F_3} (P_1 \Join P_2 ))$, where $F_3$
-- logical condition: $(\text{Stud\_ID} > 300)$. Requesting the server in this
case is not required.

% needed in second column of first page if using \IEEEpubid
%\IEEEpubidadjcol

\section{Logical constraint definition and properties}

To simplify domain computations we will consider logical formulas in
Disjunctive Normal Form (DNF). In general case formula $F$ has the following
form:

\begin{equation}
F = K_1 \vee K_2 \vee \dots \vee K_m ,
\label{def_F_1}
\end{equation}
\begin{equation}
K_i = T_1 \&\ T_2 \dots \&\ T_n, i = 1, \dots, m ,
\label{def_F_2}
\end{equation}
here $T_j, j = 1, \dots, n$ - predicates, where expanded attribute names are
specified. $R_i.A_j$ means attribute $A_j$ in relation $R_i$.
Those predicates can be:
\begin{itemize}
\item comparison operation $ \n{Expr}_1\ \theta\ \n{Expr}_2\ $, $\theta$
-- comparison operator $(\theta \in \{=, \neq, >, <, \leq, \geq\})$,
$\n{Expr}_i$ -- type conformant expressions, defined in a space of expanded
attribute names and constants;
\item operation $\n{Expr}_1\ \n{[NOT]}\ \n{BETWEEN}\ \n{Expr}_2\ \n{AND}\
\n{Expr}_3$
(symbols inside square brackets $[*]$ are optional);
\item operation $\n{Expr}\ \n{[NOT]}\ \n{IN}\ S$, here $S$ - values list or
subquery, having a set of attribute $R_i.A_j$ values as a result;
\item operation $\n{Str}_1\ \n{[NOT]}\ \n{LIKE}\ \n{Str}_2$, here $\n{Str}_i$
-- strings;
\item operation $\n{Expr}\ \theta\ \n{ALL/ANY}\ S$.
\end{itemize}

\begin{note}
We assume logical formulas have no trivial conditions on attributes, for
example, $Expr_1 = Expr_1$ and those reduced to such form. In general, we assume
that $R_i.A_l$ domain is not fully contained in
$T_j (\dots , R_i . A_l ,\dots )$ predicate truth space. Such conditions can be
removed from the formula without a change in the truth space (we will define it
later).
\label{trivial}
\end{note}

\begin{mydef}
Space of attributes contained in a formula show the dimension of the formula and
is denoted as $\<F> $.

\begin{equation}
\<F> = \{R_1^F.A_1^F, \dots, R_k^F.A_k^F\} .
\label{def_F_3}
\end{equation}
\end{mydef}


Listed variants of operations don't use every single SQL capability. For example,
we don't use the $\n{EXISTS}$ predicate, because it has no expanded attribute
names in it. The $\n{NULL}$ predicate is used for another purpose in our paper.

During logical formula domain calculation, if we have some attribute having 
$NULL$ value on a tuple $t$, we then get $\n{UNKNOWN}$ value for the whole
formula, because SQL-query results follow the Three-valued logic. It leads us to
ambiguous interpretations of results of both usual users and experienced
programmers. To solve this problem we suggest the following constraint: every
attribute in $F^{\ast}$ is supplied with a property ``Use of undefined value''
with two mutually exclusive values: ``Yes'' or ``No''. The reasoning behind this
property is the following: if it is assigned ``Yes'', then we leave tuples with
$\n{NULL}$ value for further consideration. Otherwise, having ``No'' in that
property guarantees us removing all such tuples.

Let's write expression (\ref{def_F_1}) for $F$ in the following form: $ F (\dots
, T_j , \dots )$, here $T_j$ -- predicates of expression (\ref{def_F_2}). After
the modification it will be the following: $F( \dots, T'_j , \dots )\wedge_{i,j}
(R_i .A_j \neq \n{NULL})$, here $\wedge_{i,j}(R_i .A_j \neq \n{NULL})$
-- conjunction of all $F$ attributes,for which $\n{NULL}$ is not allowed, and
$T'_j = (T_j \vee_{i,j}(R_i .A_j = \n{NULL}))$, here $\vee_{i,j}(R_i .A_j =
\n{NULL})$ -- disjunction of all $F$ attributes, for which $\n{NULL}$ is allowed.
Outer brackets for $T'_j$ predicate define operation priority. We can see that
this logical formula can only have $\n{TRUE}$ and $\n{FALSE}$ values when
considering it in Three-valued logic. We can also note that if tuples don't have
undefined values, the initial formula $F$ will be equivalent to the transformed
one, so semantics of view $P$ almost undistorted. For the disclosure of the term
"almost" we consider example. Assume $F = R_1.A_2 > 3 \vee R_3.A_4 < 4$. Let
tuple $t$ for $R_1.A_2$ has value $\n{NULL}$, with a property ``Use of undefined
value'' equal to ``Yes'', and $R_3.A_4$ value equal to 5, so value of
$R_3.A_4 < 4$ is $\n{FALSE}$. Then transform of formula $F$ on tuple $t$ will be
equal $\n{TRUE}$, which is not obvious.

Hereafter we will assume all the $F$ formulas to be transformed.

We consider set $\mathcal{A} =$ $\{(a_1, \dots, a_n) \mid a_i \in Dom(A_i),
i=1,\dots,n\}$, here $Dom(A_i)$ -- is a domain of $A_i$. Cartesian product
$Dom(A_1)\times Dom(A_2)\times \dots \times Dom(A_n)$ is an $n$-dimensional
space of all values for all Database attributes. Data constraints bound this
space to some set of points that represents a set of available Database
states.

In our example with University Study plan, the $\mathcal{A}$ can be the
following:

\begin{multline*}
\mathcal{A} = Dom(\text{Stud\_ID})\times Dom(\text{Stud\_Name}) \times\\
\times Dom(\text{Group}) \times Dom(\text{Room\_ID})
\times Dom(\text{Course}) \times\\
\times Dom(\text{Score}) = \{205, 315, 461\}\times\{\text{Rachel Davis, Noam}\\
\text{Angrist, Cameron McCord}\}
\times\{\text{M10, M11}\} \times \{100, 101,\\
102, 103\} \times \{\text{Physics, Chemistry}\} \times \{70, 80, 90\}
\end{multline*}

Obviously, without constraints this set represents more states than DB can have.
For example, each student has his/her own ID, whereas there are 3 corresponding
IDs for each student in $\mathcal{A}$: $\{205, 315, 461\}\times \{\text{Rachel
Davis, Noam Angrist, Cameron McCord}\}$.

\begin{note}
Dimension of formula $F$ can be smaller than dimension of $\mathcal{A}$. In this
case, we consider the equivalent form of the formula that has all other
attributes taking any values in their domains.
\end{note}

Looking back to our Example 1, we see that the formula in Query 1 contains only
two attributes: $F_1 = (\text{Stud\_ID} > 210\ \&\ \text{Course} =
\text{Physics})$, in other words $\<F> = 2$. The equivalent representation that
we will use is the following:

\begin{multline*}
F_1 = (\text{Stud\_ID} > 210\ \&\
\text{Course} = \text{Physics}\ \&\\
\text{Stud\_Name}\ \textit{IN}\ \{\text{Rachel Davis, Noam Angrist, Cameron}\\
\text{McCord}\}\ \&\ \text{Group}\ \textit{IN}\ \{\text{M10, M11}\}\ \&\
\text{Room\_ID}\ \textit{IN}\\
\{100, 101, 102, 103\}\ \&\ \text{Score}\ \textit{IN}\
\{70, 80, 90\})
\end{multline*}

The motivation behind this transformation is to allow us comparing truth spaces
of formulas defined on different attributes.

\begin{mydef}
Truth space of logical formula $F$, defined by (\ref{def_F_1}),
(\ref{def_F_2}), (\ref{def_F_3}), is a set $M (F) = \{a \in \mathcal{A} \mid
F(a) = \n {TRUE}\}$.
\end{mydef}

So, the truth space of our formula $F_1$ will be the following subset of
$\mathcal{A}$: $M (F) = $ \{(315, Rachel Davis, M10, 100, Physics, 70), (315,
Noam Angrist, M10, 100, Physics, 70), \dots (461, Cameron McCord, M10, 100,
Physics, 90)\}

$M (F)$ for a given formula $F$, written in DNF, is nothing but a union of Truth
spaces of distinct conjunctive clauses. Truth space of a single conjunctive
clause is an intersection of Truth spaces of it's predicates.

\begin{note}
The complexity of $\n{Expr}$, $S$ and $\n{Str}$ predicates is defined by the
software's ability to compute Truth spaces of formulas, as it is shown in
example 1.
\end{note}

\begin{mydef}
Given a logical formula $F$, defined by (\ref{def_F_1}), (\ref{def_F_2}),
(\ref{def_F_3}), the projection of $F$ on the $X$ attribute set is a logical
formula $F[X], \<F[X]> = X$, that has all its predicates with $R_i^F.A_i^F
\notin X$ replaced with trivial predicate $\n{TRUE}$.
\label{projection}
\end{mydef}

So, the projection of $F_1$ on the attribute set $X = \{\text{Stud\_ID}\}$ will be
$F_1[\text{Stud\_ID}] = (\text{Stud\_ID} > 210\ \&\ \text{TRUE}) = (\text{Stud\_ID} >
210)$ 

\begin{statement}[Inclusion property]
$\forall X \subseteq \<F> \quad M(F) \subseteq M(F[X])$
\label{proj_property_of_inclusion}
\end{statement}
\begin{proof}
If $X = \<F> $, then $F = F[X]$ and $M(F) = M(F[X])$. Assume $X \subset \<F> $.
Consider arbitrary point $a \in M(F)$, i.e. $F(a)=\n{TRUE}$. $F$ to $F[X]$
transformation is made by replacing $T_j$ predicates that contain $A_j \in
\<F> $ attribute, such that $A_j \notin X$ with $\n{TRUE}$ value. We get
$F([X])=\n{TRUE}$ according to (\ref{def_F_1}) and (\ref{def_F_2}). Therefore,
$a \in M(F[X])$.
\end{proof}

This property of logical formulas is used while building new data view from given
IDVs.

\section{Intermediate Data View properties research}

We denote saved IDVs as $P$=$\{ P_1$, $P_2$, $\dots$, $P_m \}$, here $P_v$ is an
IDV, $P_v = \pi_{X_v}(\sigma_{F_v} (R^v_1 \Join R^v_2 \Join \dots \Join
R^v_{s(v)} ))$, $s(v)$ -- amount of relations in Database that were used while
building $P_v$,
$\pi_{X_v}$ -- projection on $X_v$ attributes, $\sigma_{F_v}$ -- selection
with $F_v$ logical formula. We need to get the following view out of $P$ IDVs:
$$P^{\ast} = \pi_{X^{\ast}}(\sigma_{F^{\ast}}
(R^{\ast}_1 \Join R^{\ast}_2\Join\dots \Join R^{\ast}_l )$$
Let’s study the problem of building data view $P^{\ast}$ using existing $P_v$ IDVs.

\begin{theorem}
$P^{\ast} \subseteq \pi_{X^{\ast}} (\sigma_{F^{\ast}[X_v]} (P_{v}))$, if:
\\a) $X^{\ast} \subseteq X_{v}$
\\b) $ \{R^{v}_{1}, \ldots, R^{v}_{s(v)}\} \subseteq \{R^{\ast}_{1}, \ldots,
R^{\ast}_{l}\} $
\\c) $M (F^{\ast}) \subseteq M (F_{v}) $.
\label{th_base}
\end{theorem}


The conditions given in the aforementioned theorem guarantee the data to build
$P^{\ast}$ to be contained in $P_v$ IDV. However, there can be excess tuples that
make $F^{\ast}$
take $\n{TRUE}$ value. They appear because there are some relations in
$R^{\ast}_1 \Join R^{\ast}_2\Join \dots \Join R^{\ast}_l $ that aren't
presented in $R^v_1 \Join R^v_2 \Join \dots \Join R^v_{s(v)} $ and will be
deleted if we join those missing relations. Using Truth spaces of logical
formulas we can query DBMS to get a minimal required data set to define excess
tuples.

In the next theorem we describe the case of coinciding relation sets.

\begin{theorem}
$P^{\ast} = \pi_{X^{\ast}} (\sigma_{F^{\ast}} (P_{v}))$, if:
\\a) $X^{\ast} \subseteq X_{v}$
\\b) $\{R^{v}_{1}, \ldots, R^{v}_{s(v)}\} = \{R^{\ast}_{1}, \ldots,
R^{\ast}_{l}\}$
\\c) $M (F^{\ast}) \subseteq M (F_{v}) $
\\d) $ \<F^{\ast}> \subseteq X_{v} $.
\label{th_base_eq}
\end{theorem}
\begin{proof}
Conditions of the theorem are a special case of theorem \ref{th_base}, so
inclusion $P^{\ast} \subseteq \pi_{X^{\ast}} (\sigma_{F^{\ast}} (P_{v}))$ is
considered  to be proven. It is necessary to show that $\pi_{X^{\ast}}
(\sigma_{F^{\ast}} (P_{v})) \subseteq P^{\ast}$. Assume tuple $t_v \in
\pi_{X^{\ast}} (\sigma_{F^{\ast}} (P_{v}))$. Let's show that $t_v
\in P^{\ast}$. From definition of tuple $t_v$, it follows that there is tuple
$t' \in R^v_1 \Join R^v_2\Join\ldots \Join R^v_{s(v)}$ and $t_v = t'[X^{\ast}]$,
$F_v(t')=\n{TRUE}$, $F^{\ast}(t')=\n{TRUE}$. According to the commutativity of
natural join, $R^v_1 \Join R^v_2\Join\ldots \Join R^v_{s(v)} = R^{\ast}_1 \Join
R^{\ast}_2\Join\ldots \Join R^{\ast}_l$. Therefore, $t' \in R^{\ast}_1 \Join
R^{\ast}_2\Join\ldots \Join R^{\ast}_l$. Hence, $t_v \in P^{\ast}$. The theorem
has been proven.
\end{proof}


It is possible to use several IDVs to build a resulting representation. First,
let's consider the following simple property of natural join.

\begin{statement}
Assume $\Re_1 = R_1 \Join \dots \Join R_k$ - natural join of some $k$ relations.
Also assume $\Re_2 = R_1 \Join \dots \Join R_k \Join R_{k+1} \Join \dots \Join
R_{n}$. Then $\Re_2 [\langle \Join_{i=1}^{k} R_i \rangle] \subseteq \Re_1$
\label{join_property}
\end{statement}
\begin{proof}
It is so indeed, because joining additional relations to $\Re_1$ can only
remove some tuples that were presented there. After the projection, we will get
the same relation $\Re_1$ if no tuples were deleted or some subset otherwise. The
order of joins doesn't matter because of commutativity of this operation.
\end{proof}

%\author{Mosin Sergey, Zykin Sergey}
\def \bigcupn {\bigcup\limits_{v=1}^{n}}
\begin{theorem}
$P^{\ast} \subseteq \pi_{X^{\ast}} ( \sigma_{F^{\ast}[X]} (P_1 \Join \dots \Join P_n))$, where $X = \bigcupn X_{v}$ if:
\\a) $X^{\ast} \subseteq X$
\\b)
$ \bigcupn \{R^{v}_{1}, \ldots, R^{v}_{s(v)}\} = \{R'_{1}, \ldots, R'_{s'}\}
\subseteq
\{R^{\ast}_{1}, \ldots, R^{\ast}_{l}\} $
\\c) $M(F^{\ast}) \subseteq M(F_{v}), v = 1,\dots,n $.

\label{th_mult}
\end{theorem}
\begin{proof}
Again, we will choose an arbitrary tuple $t^{\ast} \in P^{\ast}$
and show that $t^{\ast} \in \pi_{X^{\ast}} ( \sigma_{F^{\ast}[X]}
(\Join_{v=1}^{n} P_{v}))$. By analogy to theorem \ref{th_base}, the tuples $t',
t'' = t'[\langle {\Join}_{i=1}^{s'} R'_i \rangle]$ are built.
Note that  $\bigcupn X_{v} \subseteq {\Join}_{i=1}^{s'} \langle R'_i \rangle$, hence
$t''[X^{\ast}] = t'[X^{\ast}] = t^{\ast}$.
The further reasonings also repeat being spread to all IRs.
We get $F^{\ast} [X] (t'') = \n{TRUE}$ and $F_v (t'') = \n{TRUE}, v = 1,\dots,n $.
Considering statement \ref{join_property} we have $t''[\langle \Join_{i=1}^{s(v)} R^v_i \rangle]
\in 
R^v_1 \Join \dots \Join R^v_{s(v)}, v = 1, \dots, n
\Rightarrow   
t''[X_v] \in P_v, v = 1, \dots, n
\Rightarrow
t''[\bigcupn X_{v}] \in P_1 \Join \dots \Join P_n$.
It means that $t^{\ast} = t''[X^{\ast}] \in 
\pi_{X^{\ast}} ( \sigma_{F^{\ast}[X]} (P_1 \Join \dots \Join P_n))$.
The theorem is proved.
\end{proof}


Just like in the previous case, there are some additional conditions that allow
us to get precise data presentation from IDVs.

%\author{Mosin Sergey, Zykin Sergey}
\begin{theorem}
$P^{\ast} =  \pi_{X^{\ast}} ( \sigma_{F^{\ast}} (P_1 \Join \dots \Join
P_n))$, where $X = \bigcupn X_{v}$ if:
\\a) $X^{\ast} \subseteq X$, $X_v \supseteq \<\Join_{i=1}^{s(v)} R^v_i> \cap (\bigcup\limits_{\substack{w=1\\ w \neq v}}^{n} \<\Join_{i=1}^{s(w)} R^w_i> ), v = 1,\dots,n$
\\b)
$ \bigcupn \{R^{v}_{1}, \ldots, R^{v}_{s(v)}\} = \{R'_{1}, \ldots, R'_{s'}\}
= \{R^{\ast}_{1}, \ldots, R^{\ast}_{l}\} $
\\c) $M(F^{\ast}) \subseteq M(F_{v}), v = 1,\dots,n $
\\d) $ \<F^{\ast}> \subseteq X $.
\label{th_mult_eq}
\end{theorem}


The conditions listed above aren't exceptional in any case. Many applications
require repeat of data entry with slight modifications. For example,
applications that
work with multidimensional data. Hypercube dimensions do not change often, so
that we can create new hypercubes from IDVs without frequent interaction with
the database.
% An example of a floating figure using the graphicx package.
% Note that \label must occur AFTER (or within) \caption.
% For figures, \caption should occur after the \includegraphics.
% Note that IEEEtran v1.7 and later has special internal code that
% is designed to preserve the operation of \label within \caption
% even when the captionsoff option is in effect. However, because
% of issues like this, it may be the safest practice to put all your
% \label just after \caption rather than within \caption{}.
%
% Reminder: the "draftcls" or "draftclsnofoot", not "draft", class
% option should be used if it is desired that the figures are to be
% displayed while in draft mode.
%
%\begin{figure}[!t]
%\centering
%\includegraphics[width=2.5in]{myfigure}
% where an .eps filename suffix will be assumed under latex, 
% and a .pdf suffix will be assumed for pdflatex; or what has been declared
% via \DeclareGraphicsExtensions.
%\caption{Simulation Results}
%\label{fig_sim}
%\end{figure}

% Note that IEEE CS typically puts floats only at the top, even when this
% results in a large percentage of a column being occupied by floats.
% However, the Computer Society has been known to put floats at the bottom.


% An example of a double column floating figure using two subfigures.
% (The subfig.sty package must be loaded for this to work.)
% The subfigure \label commands are set within each subfloat command, the
% \label for the overall figure must come after \caption.
% \hfil must be used as a separator to get equal spacing.
% The subfigure.sty package works much the same way, except \subfigure is
% used instead of \subfloat.
%
%\begin{figure*}[!t]
%\centerline{\subfloat[Case I]\includegraphics[width=2.5in]{subfigcase1}%
%\label{fig_first_case}}
%\hfil
%\subfloat[Case II]{\includegraphics[width=2.5in]{subfigcase2}%
%\label{fig_second_case}}}
%\caption{Simulation results}
%\label{fig_sim}
%\end{figure*}
%
% Note that often IEEE CS papers with subfigures do not employ subfigure
% captions (using the optional argument to \subfloat), but instead will
% reference/describe all of them (a), (b), etc., within the main caption.


% An example of a floating table. Note that, for IEEE style tables, the 
% \caption command should come BEFORE the table. Table text will default to
% \footnotesize as IEEE normally uses this smaller font for tables.
% The \label must come after \caption as always.
%
%\begin{table}[!t]
%% increase table row spacing, adjust to taste
%\renewcommand{\arraystretch}{1.3}
% if using array.sty, it might be a good idea to tweak the value of
% \extrarowheight as needed to properly center the text within the cells
%\caption{An Example of a Table}
%\label{table_example}
%\centering
%% Some packages, such as MDW tools, offer better commands for making tables
%% than the plain LaTeX2e tabular which is used here.
%\begin{tabular}{|c||c|}
%\hline
%One & Two\\
%\hline
%Three & Four\\
%\hline
%\end{tabular}
%\end{table}


% Note that IEEE does not put floats in the very first column - or typically
% anywhere on the first page for that matter. Also, in-text middle ("here")
% positioning is not used. Most IEEE journals use top floats exclusively.
% However, Computer Society journals sometimes do use bottom floats - bear
% this in mind when choosing appropriate optional arguments for the
% figure/table environments.
% Note that, LaTeX2e, unlike IEEE journals, places footnotes above bottom
% floats. This can be corrected via the \fnbelowfloat command of the
% stfloats package.



\section{Conclusion}
The results obtained will be used in the technology of dynamic building of
multidimensional data. In addition, they are intended to be used with graphic
processors (GPU) to perform the intermediate filtering operations and compound
of different data representations. Researches in this field have shown that the
operations performed on GPU are much faster than on CPU for small amounts of
data. Performance decreases with increasing amounts of data. The reason is
obvious: as long as the data is put in fast memory of GPU the operations with
them are performed fast. When increasing the amount of data, it is necessary to
move the data to different type of memory, which reduces productivity. However,
GPU
manufacturers are constantly expanding fast memory. At the same time, dimensions
of multidimensional data require relatively small memory, so GPUs are suitable
in solving this problem.

Multidimensional data building systems that use redundant data often face the
problem of updating it. It is often solved by the periodic updating of the
hypercube contents.
A similar method can be used to update the views $P_v$. To reduce the update
time it might be a good practice to use the change log on the database server
and update only those views that have original data changed. However, it should
be noted again that those dimensions don't change often.

% if have a single appendix:
%\appendix[Proof of the Zonklar Equations]
% or
%\appendix  % for no appendix heading
% do not use \section anymore after \appendix, only \section*
% is possibly needed

% use appendices with more than one appendix
% then use \section to start each appendix
% you must declare a \section before using any
% \subsection or using \label (\appendices by itself
% starts a section numbered zero.)
%

% use section* for acknowledgement
\ifCLASSOPTIONcompsoc
  % The Computer Society usually uses the plural form
  \section*{Acknowledgments}
\else
  % regular IEEE prefers the singular form
  \section*{Acknowledgment}
\fi

The authors would like to thank the Russian Foundation for Basic Research for
the grant \#12-07-00066-a.

% Can use something like this to put references on a page
% by themselves when using endfloat and the captionsoff option.
\ifCLASSOPTIONcaptionsoff
  \newpage
\fi



% trigger a \newpage just before the given reference
% number - used to balance the columns on the last page
% adjust value as needed - may need to be readjusted if
% the document is modified later
%\IEEEtriggeratref{8}
% The "triggered" command can be changed if desired:
%\IEEEtriggercmd{\enlargethispage{-5in}}

% references section
% can use a bibliography generated by BibTeX as a .bbl file
% BibTeX documentation can be easily obtained at:
% http://www.ctan.org/tex-archive/biblio/bibtex/contrib/doc/
% The IEEEtran BibTeX style support page is at:
% http://www.michaelshell.org/tex/ieeetran/bibtex/
\bibliographystyle{IEEEtran}
\bibliography{references}
% argument is your BibTeX string definitions and bibliography database(s)
%\bibliography{IEEEabrv,../bib/paper}
%
% <OR> manually copy in the resultant .bbl file
% set second argument of \begin to the number of references
% (used to reserve space for the reference number labels box)

%\begin{thebibliography}{2}
%\begin{thebibliography}{9}
\bibitem{zyk_pol}
\textit{Sergey Zykin, Andrey Poluyanov}
Multidimensional data building using Intermediate Representations.
"Administration problems".
2013.
NO. 5, P. 54--59.

\bibitem{chang}
\textit{Chang-Sup Park, Myoung Ho Kim b, Yoon-Joon Lee}
Usability-based caching of query results in OLAP systems.
// The Journal of Systems and Software
2003.
Vol. 68, - P. 103--119

\bibitem{baralis}
\textit{Baralis, E., Paraboschi, S., Teniente, E.}
Materialized view selection in a multidimensional database.
// Proc. of the 23rd International Conference on Very Large Data Bases,
Athens, Greece.
1997.
- P. 318--329

\bibitem{kalnis}
\textit{Kalnis, P., Papadias, D.}
Proxy-server architectures for OLAP.
// Proceedings of the 2001 ACM SIGMOD International Conference on Management of Data.
- Santa Barbara, CA, 2001.
- P. 367--378.

\bibitem{Afrati06}
\textit{Afrati F., Li C., Mitra P.}
Rewriting queries using views in the presence of arithmetic comparisons
// Theoretical Computer Science.
- 2006.
- Vol. 368.
- P. 88--123.

\bibitem{Keller96}
\textit{Keller M., Basu J.}
A predicate-based caching scheme for client-server database architectures
// VLDB Journal.
- 1996. N 5.
- P. 35--47.

\bibitem{shim}
\textit{Shim, J., Scheuermann, P., Vingralek R.}
Dynamic caching of query results for decision support systems.
// Proceedings of the 11th International Conference on Scientific and Statistical Database Management.
- Cleveland, OH, 1999.
- P. 254--263.

\end{thebibliography}
%\bibitem{IEEEhowto:kopka}
%%This is an example of a book reference
%H. Kopka and P.W. Daly, \emph{A Guide to {\LaTeX}}, third ed. Harlow, U.K.:
%Addison-Wesley, 1999.

%\bibitem{IEEEhowto:coming}
%%This is an example of a Transactions article reference
%D.S. Coming and O.G. Staadt, "Velocity-Aligned Discrete Oriented Polytopes for
%Dynamic Collision Detection," IEEE Trans. Visualization and Computer Graphics,
%vol. 14,  no. 1,  pp. 1-12,  Jan/Feb  2008, doi:10.1109/TVCG.2007.70405.

%This is an example of a article from a conference proceeding
%H. Goto, Y. Hasegawa, and M. Tanaka, "Efficient Scheduling Focusing on the Duality of MPL Representation," Proc. IEEE Symp. Computational Intelligence in Scheduling (SCIS '07), pp. 57-64, Apr. 2007, doi:10.1109/SCIS.2007.367670.

%This is an example of a PrePrint reference
%J.M.P. Martinez, R.B. Llavori, M.J.A. Cabo, and T.B. Pedersen, "Integrating Data Warehouses with Web Data: A Survey," IEEE Trans. Knowledge and Data Eng., preprint, 21 Dec. 2007, doi:10.1109/TKDE.2007.190746.

%Again, see the IEEEtrans_HOWTO.pdf for several more bibliographical examples. Also, more style examples
%can be seen at http://www.computer.org/author/style/transref.htm
%\end{thebibliography}

% biography section
% 
% If you have an EPS/PDF photo (graphicx package needed) extra braces are
% needed around the contents of the optional argument to biography to prevent
% the LaTeX parser from getting confused when it sees the complicated
% \includegraphics command within an optional argument. (You could create
% your own custom macro containing the \includegraphics command to make things
% simpler here.)
%\begin{biography}[{\includegraphics[width=1in,height=1.25in,clip,keepaspectratio]{mshell}}]{Michael Shell}
% or if you just want to reserve a space for a photo:

%\begin{IEEEbiography}{Michael Shell}
%Biography text here.
%\end{IEEEbiography}

\newpage

% if you will not have a photo at all:
\begin{IEEEbiographynophoto}{Sergey Vladimirovich Mosin}
is a PhD student in the Sobolev Institute of Mathematics of the Siberian Branch
of the Russian Academy of Sciences. His
research interests include Databases, Data analysis, Algorithm Theory and
Internet technologies. Mosin received a Specialist degree in Applied Mathematics
and Computer Science from Omsk F. M. Dostoevsky State University. Contact him at
svmosin@gmail.com
\end{IEEEbiographynophoto}

% insert where needed to balance the two columns on the last page with
% biographies
%\newpage

\begin{IEEEbiographynophoto}{Sergey Vladimirovich Zykin}
is a professor, doctor of science (informatics) with a focus on theory of
Databases designing and head of the laboratory at the Sobolev Institute of
Mathematics of the Siberian Branch of the Russian Academy of Sciences. He has
published some 70 scientific papers as conference-, journal-, and book
contributions. Contact him at szykin@mail.ru
\end{IEEEbiographynophoto}

% You can push biographies down or up by placing
% a \vfill before or after them. The appropriate
% use of \vfill depends on what kind of text is
% on the last page and whether or not the columns
% are being equalized.

\vfill

% Can be used to pull up biographies so that the bottom of the last one
% is flush with the other column.
%\enlargethispage{-5in}



% that's all folks
\end{document}



