\begin{theorem}
$P^{\ast} = \pi_{X^{\ast}} (P_{v})$, if:
\\a) $X^{\ast} \subseteq X_{v}$
\\b) $\{R^{v}_{1}, \ldots, R^{v}_{s(v)}\} = \{R^{\ast}_{1}, \ldots,
R^{\ast}_{l}\}$
\\c) $M (F^{\ast}) = M (F_{v}) $.
\label{th_base_eq}
\end{theorem}
\begin{proof}
Conditions of the theorem are a special case of theorem \ref{th_base}, so
inclusion $P^{\ast} \subseteq \pi_{X^{\ast}} (P_{v})$ is considered  to be
proven. It is necessary to show that $\pi_{X^{\ast}} (P_{v}) \subseteq
P^{\ast}$. Assume tuple $t_v \in \pi_{X^{\ast}} (P_{v})$. Let's show that $t_v
\in P^{\ast}$. From definition of tuple $t_v$, it follows that there is tuple
$t' \in R^v_1 \Join R^v_2\Join\ldots \Join R^v_{s(v)}$ and $t_v = t'[X^{\ast}]$,
$F_v(t')=\n{TRUE}$. According to the commutativity of natural join, $R^v_1 \Join
R^v_2\Join\ldots \Join R^v_{s(v)} = R^{\ast}_1 \Join R^{\ast}_2\Join\ldots \Join
R^{\ast}_l$. Therefore, $t' \in R^{\ast}_1 \Join R^{\ast}_2\Join\ldots \Join
R^{\ast}_l$. From condition c) we have $F^{\ast} (t')=\n{TRUE}$. Hence, $t_v \in
P^{\ast}$. The theorem has been proven.
\end{proof}
