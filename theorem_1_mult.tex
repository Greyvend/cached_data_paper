\author{Мосин Сергей, Зыкин Сергей}
\def \bigcupn {\bigcup\limits_{v=1}^{n}}
\begin{theorem}
$P^{\ast} \subseteq \pi_{X^{\ast}} ( \sigma_{F^{\ast}[X]} (P_1 \bowtie \dots \bowtie P_n))$, где $X = \bigcupn X_{v}$ если:
\\а) $X^{\ast} \subseteq X$
\\б)
$ \bigcupn \{R^{v}_{1}, \ldots, R^{v}_{s(v)}\} = \{R'_{1}, \ldots, R'_{s'}\}
\subseteq
\{R^{\ast}_{1}, \ldots, R^{\ast}_{l}\} $
\\в) $\<F_v> \subseteq \<F^{\ast}> , M (F^{\ast}[\<F_v> ]) \subseteq M (F_{v}), v = 1,\dots,n $.

\label{th_mult}
\end{theorem}
\begin{proof}
Аналогично, выберем некий кортеж $t^{\ast} \in P^{\ast}$ и покажем, 
что $t^{\ast} \in \pi_{X^{\ast}} ( \sigma_{F^{\ast}[X]} (\bowtie_{v=1}^{n} P_{v}))$. Строим кортежи
$t', t'' = t'[\langle {\bowtie}_{i=1}^{s'} R'_i \rangle]$ таким же образом,
как в Теореме \ref{th_base}. Заметим, что $\bigcupn X_{v} \subseteq 
{\bowtie}_{i=1}^{s'} \langle R'_i \rangle$, следовательно, получим
$t''[X^{\ast}] = t'[X^{\ast}] = t^{\ast}$.
Дальнейшие рассуждения также повторяются, распространяясь на все промежуточные представления. Получаем $F^{\ast} [X] (t'') = \n{TRUE}$ и $F_v (t'') = \n{TRUE}, v = 1,\dots,n $.
По утверждению \ref{join_property} $t''[\langle \bowtie_{i=1}^{s(v)} R^v_i \rangle]
\in 
R^v_1 \bowtie \dots \bowtie R^v_{s(v)}, v = 1, \dots, n
\Rightarrow   
t''[X_v] \in P_v, v = 1, \dots, n
\Rightarrow
t''[\bigcupn X_{v}] \in P_1 \bowtie \dots \bowtie P_n$.
А это значит, что $t^{\ast} = t''[X^{\ast}] \in 
\pi_{X^{\ast}} ( \sigma_{F^{\ast}[X]} (P_1 \bowtie \dots \bowtie P_n))$.
Теорема доказана.
\end{proof}
