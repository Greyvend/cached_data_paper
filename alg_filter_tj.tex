Следующий алгоритм осуществляет операцию естественного соединения $\sigma_{F^{\ast}[X_v]}(P_v)$ с недостающими в $\{R^v_1, \dots, R^v_{s(v)}\} $ отношениями. Во избежание проведения лишних декартовых произведений также восстанавливаются некоторые отсеянные проекцией атрибуты, которые входят в множество атрибутов недостающих отношений. Затем проверяется полная формула $F^{\ast}$ для устранения оставшихся лишних кортежей.

Обозначим $\mathbf{R^v} = \{R^v_1, \dots, R^v_{s(v)}\} $.
$\{\overline{R}_{1},\dots,\overline{R}_{m}\} = \{R^{\ast}_1,\dots,R^{\ast}_l\}
\setminus \mathbf{R^v}$.
\begin{algorithm}[H]
\caption{Одномерный случай}
\begin{algorithmic}
\REQUIRE
$P_v$,
$\{R^v_1,\dots,R^v_{s(v)}\}$,
$X_v$,
$\overline{\Re} = \overline{R}_1 \bowtie \overline{R}_2\bowtie\ldots\bowtie \overline{R}_m $,
$Y = \<\overline{\Re}> $
$F^{\ast}$,
$X^{\ast}$
\ENSURE $R = P^{\ast}$
\STATE $R \leftarrow \sigma_{F^{\ast}[X_v]}(P_v)$

\WHILE {$R \nsupseteq (Y \cup \<F^{\ast}> )$}
   \FORALL{$R_{cur} \in \mathbf{R^v}$}
      \IF{$R_{cur} \cap R \cap (Y \cup \<F^{\ast} > ) \neq \emptyset$}
         \STATE $ R \leftarrow R \bowtie R_{cur} $
         \STATE убрать $R_{cur}$ из дальнейшего рассмотрения
      \ENDIF
   \ENDFOR
   
   \IF{не добавлено ни одного отношения после последнего просмотра}
     \FORALL{$R_{cur} \in \mathbf{R^v}$}
         \IF{$R_{cur} \cap (Y \cup \<F^{\ast} > ) \neq \emptyset$}
            \STATE $ R \leftarrow R \bowtie R_{cur} $
            \STATE $\mathbf{R^v} \leftarrow \mathbf{R^v} \setminus R_{cur}$
         \ENDIF
      \ENDFOR
   \ENDIF
\ENDWHILE

\STATE $R \leftarrow R \bowtie \overline{\Re}$
\STATE $R \leftarrow \pi_{X^{\ast}}(\sigma_{F^{\ast}}(R))$
\end{algorithmic}
\end{algorithm}

Многомерный случай аналогичен одномерному. Заметим лишь, что сперва проводится соединение набора промежуточных представлений, а только затем проверяется формула $F^{\ast}[X]$ всвязи с тем, что необходимо обеспечить полный набор указанных в формуле атрибутов.
Для алгоритма, описывающего многомерный случай, введем следующие обозначения:
$\mathbf{P} = \{P_1, \dots, P_n\} $.
$\mathbf{R'} = \{R_1', \dots, R_s'\} $.
$\{\overline{R}_1,\dots,\overline{R}_m\} = \{R^{\ast}_1,\dots,R^{\ast}_l\}
\setminus \mathbf{R'} $.
\begin{algorithm}[H]
\caption{Многомерный случай}
\begin{algorithmic}
\REQUIRE
$\mathbf{P}$,
$\mathbf{R'}$,
$X$,
$\overline{\Re} = \overline{R}_1 \bowtie \overline{R}_2\bowtie\ldots\bowtie
\overline{R}_m $,
$Y = \<\overline{\Re}> $
$F^{\ast}$,
$X^{\ast}$
\ENSURE $R = P^{\ast}$
\STATE $\mathbf{R_{pool}} \leftarrow \mathbf{R'} \cup \mathbf{P} $
\STATE $R \leftarrow P_1$
\STATE $\mathbf{R_{pool}} \leftarrow \mathbf{R_{pool}} \setminus P_1$

\WHILE {$R \nsupseteq (Y \cup \<F^{\ast}> )$}
   \FORALL{$R_{cur} \in \mathbf{R_{pool}}$}
      \IF{$R_{cur} \cap R \cap (Y \cup \<F^{\ast} > ) \neq \emptyset$}
         \STATE $ R \leftarrow R \bowtie R_{cur}$
         \STATE $\mathbf{R_{pool}} \leftarrow \mathbf{R_{pool}} \setminus
           R_{cur}$
      \ENDIF
   \ENDFOR
   
   \IF{не добавлено ни одного отношения после последнего просмотра}
      \FORALL{$R_{cur} \in \mathbf{R_{pool}}$}
         \IF{$R_{cur} \cap (Y \cup \<F^{\ast}> ) \neq \emptyset$ or $R_{cur} \in
           \mathbf{P}$}
            \STATE $ R \leftarrow R \bowtie R_i $
            \STATE $\mathbf{R_{pool}} \leftarrow \mathbf{R_{pool}} \setminus
           R_{cur}$
         \ENDIF
      \ENDFOR  
   \ENDIF
\ENDWHILE

\STATE $R \leftarrow \sigma_{F^{\ast}[X]}(R)$
\STATE $R \leftarrow R \bowtie \overline{\Re}$
\STATE $R \leftarrow \pi_{X^{\ast}}(\sigma_{F^{\ast}}(R))$
\end{algorithmic}
\end{algorithm} 
