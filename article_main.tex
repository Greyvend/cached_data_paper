\documentclass[10pt,a4paper]{article}
\author{Мосин Сергей, Зыкин Сергей}
\usepackage{amssymb,amsmath}
\usepackage[utf8]{inputenc}
\usepackage[russian]{babel}
\usepackage{amsthm}
\usepackage{amsfonts}
\usepackage{algorithmic}
\usepackage{algorithm}
\usepackage{xcolor}
\floatname{algorithm}{Алгоритм}
\newtheorem{theorem}{Теорема}
\newtheorem{mydef}{Определение}
\newtheorem{statement}{Утверждение}
\newtheorem*{consequence}{Следствие}
\newtheorem{note}{Замечание}
\def \eval #1#2{\left.#1\right\vert_{#2}}
\def \<#1> {\langle #1 \rangle}
\def \n #1{\mathit{#1}}
\begin{document}

\section{Введение}
Многие современные решения в области анализа больших объемов данных
используют технологии аналитической обрабоки данных (Online Analytical
Processing), сформировавшиеся еще в начале 90-х годов \cite{codd}. За последнее
время в этой области было получено большое количество фундаментальных
\cite{lecht,lehner,mazon} и практических результатов \cite{vassi, peder,
progressive, giorg}.

Целью данной работы является изучение проблемы динамического формирования
гиперкубического представления базы данных из реляционной базы данных (РБД) с
помощью зарезервированных представлений данных на компьютере, оснащенном
графическим процессором (GPU). Эта проблема связана с областью оптимизации
запросов, поскольку нацелена на сокращение объема передаваемых данных с сервера
БД. Зарезервированные данные активно используется в системах управления базами
данных (СУБД). Но в большинстве случаев это касается повторного использования
данных, предварительно записанных в буферный пул, без предварительного анализа
содержимого на предмет возможности частичного или комбинированного
использования. Работа СУБД ограничивается тем, что при выполнении очередного
запроса блоки данных не запрашиваются с внешних устройств, если они есть в
буферном пуле, т. е. анализируются номера блоков, а не их содержимое.

В области теоретических исследований ситуация складывается следующим образом.
Огромное число публикаций посвящено проблеме построения оптимального плана
запроса на основе формальных правил, в которых не используются области
определения предикатов в SQL-операторах (логическая оптимизация) либо эти
области учитываются при вычислении статистических оценок для оптимизации
физического доступа к БД. Близкими по методам решения являются задачи
выполнения запросов на потоках данных [9, 10], однако различные, по сравнению
с настоящей работой, цели приводят к различным результатам. Наиболее близка к
рассматриваемой проблеме работа [11]. В ней рассматриваются конъюнктивные
запросы над доменами данных с предикатами в виде арифметических сравнений и
представлены алгоритмы вычисления запросов с использованием существующих
представлений. В настоящей работе рассматривается специальный вид универсального
реляционного запроса над отношениями базы данных, а не над отдельными доменами.
Хотя цели в обеих работах совпадают, результаты различны по указанной причине.
В частности, в настоящей работе нет необходимости разрабатывать алгоритмы, так
как их замещает реляционная алгебра.

Существует ряд работ, в которых рассматривается подход к оптимизации запросов
именно в многомерных базах данных и хранилищах данных.
Есть два основных подхода для резервирования результатов запросов: статический
\cite{baralis, gupta, gupta-mumick} и динамический\cite{scheuermann, shim,
kalnis, chang}. Первый базируется на использовании набора фиксированных
запросов, во втором же предполагается динамический выбор результатов запросов
для резервирования на основе статистики появления а также вычислительной
стоимости выполнения запросов. В отличие от подходов, применяемых в реляционных
СУБД, здесь проводится анализ самих запросов. Однако подходы, описанные в
данных статьях, применимы лишь для технологий реляционных многомерных баз данных
(ROLAP) и используют специальные для многомерных баз данных типы запросов, в то
время как в нашем случае используются стандартные SQL запросы. Данная работа
базируется на результатах, полученных в статье \cite{zyk_pol}. Устранено
условие, ограничивающее атрибуты промежуточного представления, а также сделано
обобщение теорем на случай нескольких промежуточных представлений. В заключение
приводятся алгоритмы, позволяющие применять на практике полученные результаты.

\section{Определение и свойства логических ограничений}

Будем рассматривать логические формулы в дизъюнктивной нормальной форме. Тогда
формула $F$ имеет вид 

\begin{equation}
F = K_1 \vee K_2 \vee \dots \vee K_m
\label{def_F_1}
\end{equation}
\begin{equation}
K_i = T_1 \& T_2 \dots \& T_n, i = 1, \dots, m
\label{def_F_2}
\end{equation}

$T_j, j = 1, \dots, n$ - предикаты, в которых явным образом
специфицированы расширенные имена атрибутов $R_i.A_j$ (атрибут $A_j$ в отношении
$R_i$):
\begin{itemize}
  \item	операция сравнения $ \n{Expr}_1 \theta \n{Expr}_2$, $\theta$ – операция
  сравнения $(\theta \in \{=, \neq, >, <, \leq, \geq\})$, $\n{Expr}_i$ –
  согласованные по типам допустимые выражения, определенные на множестве
  расширенных имен атрибутов и констант;
  \item операция $\n{Expr}_1 \n{[NOT]} \n{BETWEEN} \n{Expr}_2 \n{AND}
  \n{Expr}_3$ (содержимое в прямоугольных скобках $[*]$ для предиката не
  является обязательным при написании);
  \item операция $\n{Expr} \n{[NOT]} \n{IN} S$, где $S$ – список значений либо
  подзапрос, результатом которого является столбец атрибута $A_j$ в отношении
  $R_i$;
  \item операция $\n{Str}_1 \n{[NOT]} \n{LIKE} \n{Str}_2$, где $\n{Str}_i$ –
  строки;
  \item операция $\n{Expr} \theta \n{ALL/ANY} S$.
\end{itemize}

\begin{note}
Здесь и далее будем полагать, что в формулах нет тривиальных условий на
атрибуты, то есть условий вида $Expr_1 = Expr_1$ и сводящихся к ним. Такие
условия могут быть удалены из формулы, не меняя области истинности (будет
определено далее).
\label{trivial}
\end{note}

\begin{mydef}
Множество атрибутов, входящих в формулу, выражает размерность формулы и
обозначается $\<F> $.

\begin{equation}
\<F> = \{R_1^F.A_1^F, \dots, R_k^F.A_k^F\}
\label{def_F_3}
\end{equation}
\end{mydef}



Перечисленные варианты операций используют не все возможности языка SQL.
Например, предикат $\n{EXISTS}$ не используется, поскольку в нем явно не
специфицированы расширенные имена атрибутов, предикат $\n{NULL}$ используется
в данной работе для другой цели.

При вычислении логического выражения может быть получено значение $\n{UNKNOWN}$,
если на текущем кортеже $t$ атрибут принимает значение $\n{NULL}$, поскольку
результаты вычисления логических выражений в SQL-запросах соответствуют
трехзначной логике. Для избежания такой ситуации формулы преобразуются
определенным образом, изложенным в статье \cite{zyk_pol}.

В дальнейшем будем использовать множество $\mathcal{A} =$ $\{(a_1, \dots, a_l)
\mid a_i \in Dom(A_i), i=1,\dots,l\}$, $l$-мерное пространство значений всех
атрибутов базы данных.

\begin{mydef}
Областью истинности логической формулы $F$, заданной (\ref{def_F_1}),
(\ref{def_F_2}), (\ref{def_F_3}), является множество, определяемое по следующему
правилу: $M (F) = \{a \in \mathcal{A} \mid F(a) = \n {TRUE}\}$.
\end{mydef}

\begin{note}
Размерность формулы $F$ может быть меньше
размерности $\mathcal{A}$, при этом атрибуты, не входящие в формулу, могут
принимать любые значения.
\end{note}

\begin{note}
Сложность компонентов предикатов $\n{Expr}$, $S$ и $\n{Str}$ определяется
возможностями программного обеспечения по вычислению областей определения
$\n{Dom} (F_v)$, что необходимо для вычисления $TJ(C_s)$ из промежуточных
представлений $P_v$.
\end{note}

В соответствии с данными определениями легко понять, как будут устроены операции
над областями, соответствующими формулам. $\n{Dom} (F)$ для некоторой формулы
$F$, заданной своей ДНФ, является объединением областей, представленных
отдельными конъюнктами формулы. Область $M(F)$ является подобластью области
$\n{Dom} (F)$, то есть $M(F) \subseteq \n{Dom} (F)$. Включение $M(F_1) \subseteq
M(F_2)$ означает, что область истинности каждого конъюнкта $F_1$ является
подобластью некоторого объединения областей истинности конъюнктов $F_2$.

Теперь введем определения, касающиеся модификации вхождения атрибутов в
логические формулы.

\begin{mydef}
Проекцей логической формулы $F$, заданной (\ref{def_F_1}), (\ref{def_F_2}),
(\ref{def_F_3}), на множество атрибутов $X$ называется логическая
формула $F[X], \<F[X]> $
$= X$, в которой все термы, содержащие
атрибуты $R_i^F.A_i^F \notin X$, заменены на тривиальный терм TRUE.
\label{projection}
\end{mydef}

\begin{statement}[Свойство включения]
$\forall X \subseteq \<F> \quad M(F) \subseteq M(F[X])$
\label{proj_property_of_inclusion}
\end{statement}
\begin{proof}
Рассмотрим некоторую точку $a \in M(F)$. 
Так как формула находится в ДНФ, то замена некоторого предиката $T$ на значение
$\n{TRUE}$, очевидно, может либо полностью сделать значение формулы $\n{TRUE}$,
если предикат $T$ единственный в конъюнкте $K_i = T$, либо сократить количество
условий в нём, если $K_i = T \& T'_1 \dots \& T'_n$. В любом из этих случаев
количество условий в формуле $F[X]$ лишь уменьшается, что гарантирует выполнение
$F[X](a) = \n{TRUE}$ и, следовательно, $a \in M(F[X])$
\end{proof}

\begin{statement}
$M(F_1) \subseteq M(F_2) \Rightarrow \<F_2> \subseteq \<F_1> $
\label{attibute_inclusion_property}
\end{statement}
\begin{proof}
Пусть это условие не выполняется. Тогда $U' = \<F_2> \setminus \<F_1> \neq
\emptyset$. Выберем точку $a = (a_1, \dots, a_l) \in M(F_1)$. Для некоторого
атрибута $A_i \in U'$ выберем соответствующую координату $a_i \in Dom(A_i)$. Так
как по замечанию \ref{trivial} в формулах нет тривиальных условий на атрибуты, то $\exists a_i'
\in Dom(A_i) : a' = (a_1, \dots, a_{i - 1}, a_i', a_{i + 1}, \dots, a_l) \notin
M(F_2)$. Однако $a' \in M(F_1)$, так как атрибут $A_i$ отсутствует в формуле
$F_1$. Следовательно, $M(F_1) \nsubseteq M(F_2)$. Полученное противоречие
доказывает утверждение.
\end{proof}

\begin{consequence}
$M(F_1) = M(F_2) \Rightarrow \<F_2> = \<F_1> $
\label{attibute_equality_property}
\end{consequence}
\begin{proof}
Пусть это условие не выполняется. Тогда, в силу предыдущего утверждения,
существует множество $U' = \<F_1> \setminus \<F_2> \neq \emptyset$. Аналогично
предыдущему доказательству, выбираем точку из любого из множеств и меняем
значение координаты, соответствующей атрибуту из множества $U'$ на такое,
которое не принаджежит множеству $M(F_1)$. Полученная точка по-прежнему
принадлежит $M(F_2)$. Мы снова пришли к противоречию. Утверждение доказано.
\end{proof}

\section{Исследование свойств промежуточных представлений данных}

\begin{theorem}
$P^{\ast} \subseteq \pi_{X^{\ast}} (\sigma_{F^{\ast}[X_v]} (P_{v}))$
, if:
\\a) $X^{\ast} \subseteq X_{v}$
\\b) 
$ \{R^{v}_{1}, \ldots, R^{v}_{s(v)}\}
\subseteq
\{R^{\ast}_{1}, \ldots, R^{\ast}_{l}\} $
\\c) $M (F^{\ast}) \subseteq M (F_{v}) $.
\label{th_base}
\end{theorem} 
\begin{proof}
Let $t^{\ast}$ be any tuple in $P^{\ast}$. We need to show that
$t^{\ast} \in \pi_{X^{\ast}} (P_{v})$. From condition  $t^{\ast} \in P^{\ast}$,
there is the tuple $t' \in R^{\ast}_1 \Join R^{\ast}_2\Join\ldots
\Join R^{\ast}_l$ and $t^{\ast} = t'[X^{\ast}]$. There may be several such tuples. We will choose the one that satisfies $F^{\ast}(t')=TRUE$.
This tuple exists indeed, because otherwise $t^{\ast}$ would not be in $P^{\ast}$. Thus, there are tuples
$t^{\ast}_i \in R^{\ast}_i$:

\begin{equation}
t^{\ast}_i = t'[\langle R^{\ast}_i\rangle], i = 1,\dots,l,
\label{eval_eq_1}
\end{equation}
\def \intersecij {\langle R^{\ast}_i \rangle \cap \langle R^{\ast}_j \rangle}
and for any pair $i$ and $j$, such that  $\intersecij \neq \emptyset$, equality is held:
\begin{equation}
t^{\ast}_i[\intersecij] = t'_i[\intersecij], i,j = 1,\dots,l,
\label{eval_eq_2}
\end{equation}

Whereas conditions (\ref{eval_eq_1}) and (\ref{eval_eq_2}) are held for the
whole set of relations $\{R^{\ast}_{1}, \ldots, R^{\ast}_{l}\}$, they are
true for any its subset also, including $\{R^{v}_{1}, \ldots, R^{v}_{s(v)}\}$.
Hence, after joining tuples $t^{\ast}_i$ from relations $R^{v}_j$ we get a tuple
$t''$, such that $t'' = t'[\langle {\Join}_{i=1}^{s(v)} R^v_i \rangle]$.
Equality $t''[X^{\ast}] = t'[X^{\ast}] = t^{\ast}$ follows from a) and $X_v
\subseteq \langle {\Join}_{i=1}^{s(v)} R^v_i \rangle$.

Condition  $F^{\ast}(t') = \n{TRUE}$ and statement 
\ref{proj_property_of_inclusion} imply the truth of the projection $F^{\ast}
[X^{\ast}]$ on a tuple $t'$, and consequently on $t''$, as the formula is defined at common attributes of these tuples. Furthermore, according to condition 3, we have $F_v (t') = \n{TRUE} \Rightarrow F_v (t'') = \n{TRUE}$.
It means $t^{\ast} = t''[X^{\ast}] \in \pi_{X^{\ast}} (P_{v})$. The theorem is
proved.
\end{proof}


\begin{theorem}
$P^{\ast} = \pi_{X^{\ast}} (P_{v})$, if:
\\a) $X^{\ast} \subseteq X_{v}$
\\b) $\{R^{v}_{1}, \ldots, R^{v}_{s(v)}\} = \{R^{\ast}_{1}, \ldots,
R^{\ast}_{l}\}$
\\c) $M (F^{\ast}) = M (F_{v}) $.
\label{th_base_eq}
\end{theorem}
\begin{proof}
Conditions of the theorem are a special case of theorem \ref{th_base}, so
inclusion $P^{\ast} \subseteq \pi_{X^{\ast}} (P_{v})$ is considered  to be
proven. It is necessary to show that $\pi_{X^{\ast}} (P_{v}) \subseteq
P^{\ast}$. Assume tuple $t_v \in \pi_{X^{\ast}} (P_{v})$. Let's show that $t_v
\in P^{\ast}$. From definition of tuple $t_v$, it follows that there is tuple
$t' \in R^v_1 \Join R^v_2\Join\ldots \Join R^v_{s(v)}$ and $t_v = t'[X^{\ast}]$,
$F_v(t')=\n{TRUE}$. According to the commutativity of natural join, $R^v_1 \Join
R^v_2\Join\ldots \Join R^v_{s(v)} = R^{\ast}_1 \Join R^{\ast}_2\Join\ldots \Join
R^{\ast}_l$. Therefore, $t' \in R^{\ast}_1 \Join R^{\ast}_2\Join\ldots \Join
R^{\ast}_l$. From condition c) we have $F^{\ast} (t')=\n{TRUE}$. Hence, $t_v \in
P^{\ast}$. The theorem has been proven.
\end{proof}


Эти результаты можно обобщить, взяв в рассмотрение целый набор промежуточных
представлений $P_v$. Но сперва приведем один факт, связанный со свойствами
естественного соединения.

\begin{statement}
Пусть $\Re_1 = R_1 \bowtie \dots \bowtie R_k$ - результат естественного
соединения некоторых $k$ отношений. Пусть также $\Re_2 = R_1 \bowtie \dots
\bowtie R_k \bowtie R_{k+1} \bowtie \dots \bowtie R_{n}$. 
Тогда $\Re_2 [\langle \bowtie_{i=1}^{k} R_i \rangle] \subseteq \Re_1$
\label{join_property}
\end{statement}

Это действительно так, ведь после проведения операции естественного соединения
с дополнительными отношениями исходные кортежи, содержавшиеся в $\Re_1$, могут
лишь ''отсеяться'' операцией естественного соединения. Тогда, взяв вырезку по
исходным атрибутам, получим как максимум то же самое отношение $\Re_1$.

\author{Мосин Сергей, Зыкин Сергей}
\def \bigcupn {\bigcup\limits_{v=1}^{n}}
\begin{theorem}
$P^{\ast} \subseteq \pi_{X^{\ast}} ( \sigma_{F^{\ast}[X]} (P_1 \bowtie \dots \bowtie P_n))$, где $X = \bigcupn X_{v}$ если:
\\а) $X^{\ast} \subseteq X$
\\б)
$ \bigcupn \{R^{v}_{1}, \ldots, R^{v}_{s(v)}\} = \{R'_{1}, \ldots, R'_{s'}\}
\subseteq
\{R^{\ast}_{1}, \ldots, R^{\ast}_{l}\} $
\\в) $M(F^{\ast}) \subseteq M(F_{v}), v = 1,\dots,n $.

\label{th_mult}
\end{theorem}
\begin{proof}
Аналогично, выберем некий кортеж $t^{\ast} \in P^{\ast}$ и покажем, 
что $t^{\ast} \in \pi_{X^{\ast}} ( \sigma_{F^{\ast}[X]} (\bowtie_{v=1}^{n} P_{v}))$. Строим кортежи
$t', t'' = t'[\langle {\bowtie}_{i=1}^{s'} R'_i \rangle]$ таким же образом,
как в Теореме \ref{th_base}. Заметим, что $\bigcupn X_{v} \subseteq 
{\bowtie}_{i=1}^{s'} \langle R'_i \rangle$, следовательно, получим
$t''[X^{\ast}] = t'[X^{\ast}] = t^{\ast}$.
Дальнейшие рассуждения также повторяются, распространяясь на все промежуточные представления. Получаем $F^{\ast} [X] (t'') = \n{TRUE}$ и $F_v (t'') = \n{TRUE}, v = 1,\dots,n $.
По утверждению \ref{join_property} $t''[\langle \bowtie_{i=1}^{s(v)} R^v_i \rangle]
\in 
R^v_1 \bowtie \dots \bowtie R^v_{s(v)}, v = 1, \dots, n
\Rightarrow   
t''[X_v] \in P_v, v = 1, \dots, n
\Rightarrow
t''[\bigcupn X_{v}] \in P_1 \bowtie \dots \bowtie P_n$.
А это значит, что $t^{\ast} = t''[X^{\ast}] \in 
\pi_{X^{\ast}} ( \sigma_{F^{\ast}[X]} (P_1 \bowtie \dots \bowtie P_n))$.
Теорема доказана.
\end{proof}


%\author{Mosin Sergey, Zykin Sergey}
\begin{theorem}
$P^{\ast} = \pi_{X^{\ast}} (P_1 \Join \dots \Join 
P_n)$, where $X = \bigcupn X_{v}$ if:
\\a) $X^{\ast} \subseteq X$, $X_v \supseteq \<\Join_{i=1}^{s(v)} R^v_i> \cap (\bigcup\limits_{\substack{w=1\\ w \neq v}}^{n} \<\Join_{i=1}^{s(w)} R^w_i> ), v = 1,\dots,n$
\\b)
$ \bigcupn \{R^{v}_{1}, \ldots, R^{v}_{s(v)}\} = \{R'_{1}, \ldots, R'_{s'}\}
= \{R^{\ast}_{1}, \ldots, R^{\ast}_{l}\} $
\\c) $M(F^{\ast}) = M(F_{1} \& \dots \& F_{n})$.
\label{th_mult_eq}
\end{theorem} 
\begin{proof}
Conditions of the theorem are a special case of theorem \ref{th_base}, so
inclusion  $P^{\ast} \subseteq \pi_{X^{\ast}} (\Join_{v=1}^{n} P_{v})$ is
considered to be proved. It is necessary to show, that $\pi_{X^{\ast}}
(\Join_{v=1}^{n} P_{v}) \subseteq P^{\ast}$. Assume tuple $t \in \pi_{X^{\ast}}
(\Join_{v=1}^{n} P_{v})$. Let's show, that $t \in P^{\ast}$. Using the property
of the projection operation and condition a) we have $\pi_{X^{\ast}}(\pi_{X_1}
(R^1_1 \Join \dots \Join R^1_{s(1)}) \Join \dots \Join \pi_{X_n}(R^n_1 \Join
\dots \Join R^n_{s(n)})) = \pi_{X^{\ast}} (R'_1 \Join \dots \Join R'_{s'})$.
Thus, according to the definition of tuple $t$, there is a tuple  $t' \in R'_1
\Join \ldots \Join R'_{s'}$ and $t = t'[X^{\ast}]$, $F_v(t')=\n{TRUE}, v =
1,\dots,n$. According to commutativity of natural join, $R'_1 \Join \ldots \Join
R'_s = R^{\ast}_1 \Join \ldots \Join R^{\ast}_l$, hence, $t' \in R^{\ast}_1 \Join
\ldots \Join R^{\ast}_l$. From condition c) we have $F^{\ast} (t')=\n{TRUE}$. Therefore $t \in P^{\ast}$. The theorem is proved.
\end{proof} 


\section{Удаление лишних кортежей из промежуточных представлений}

Результаты, полученные в теоремах \ref{th_base}, \ref{th_mult} нельзя сразу
использовать на практике. Отношение $P^{\ast}$ содержится в имеющемся
представлении, однако неизвестно, какие именно кортежи являются частью
$P^{\ast}$. Лишние кортежи могут возникнуть по двум причинам:
\begin{enumerate}
  \item В силу Утверждения \ref{proj_property_of_inclusion}, так как правая
  часть содержит проекцию исходной формулы $F^{\ast}$ на некоторое множество
  атрибутов.
  \item В силу Утверждения \ref{join_property}, так как в промежуточном
  представлении отношений не больше, чем в искомом $P^{\ast}$.
\end{enumerate}
Поэтому необходимо сперва удалить лишние кортежи (если они есть) из
промежуточных представлений.

Следующий алгоритм осуществляет операцию естественного соединения $\sigma_{F^{\ast}[X_v]}(P_v)$ с недостающими в $\{R^v_1, \dots, R^v_{s(v)}\} $ отношениями. Во избежание проведения лишних декартовых произведений также восстанавливаются некоторые отсеянные проекцией атрибуты, которые входят в множество атрибутов недостающих отношений. Затем проверяется полная формула $F^{\ast}$ для устранения оставшихся лишних кортежей.

Обозначим $\mathbf{R^v} = \{R^v_1, \dots, R^v_{s(v)}\} $.
$\{\overline{R}_{1},\dots,\overline{R}_{m}\} = \{R^{\ast}_1,\dots,R^{\ast}_l\}
\setminus \mathbf{R^v}$.
\begin{algorithm}[H]
\caption{Одномерный случай}
\begin{algorithmic}
\REQUIRE
$P_v$,
$\{R^v_1,\dots,R^v_{s(v)}\}$,
$X_v$,
$\overline{\Re} = \overline{R}_1 \bowtie \overline{R}_2\bowtie\ldots\bowtie \overline{R}_m $,
$Y = \<\overline{\Re}> $
$F^{\ast}$,
$X^{\ast}$
\ENSURE $R = P^{\ast}$
\STATE $R \leftarrow \sigma_{F^{\ast}[X_v]}(P_v)$

\WHILE {$R \nsupseteq (Y \cup \<F^{\ast}> )$}
   \FORALL{$R_{cur} \in \mathbf{R^v}$}
      \IF{$R_{cur} \cap R \cap (Y \cup \<F^{\ast} > ) \neq \emptyset$}
         \STATE $ R \leftarrow R \bowtie R_{cur} $
         \STATE убрать $R_{cur}$ из дальнейшего рассмотрения
      \ENDIF
   \ENDFOR
   
   \IF{не добавлено ни одного отношения после последнего просмотра}
     \FORALL{$R_{cur} \in \mathbf{R^v}$}
         \IF{$R_{cur} \cap (Y \cup \<F^{\ast} > ) \neq \emptyset$}
            \STATE $ R \leftarrow R \bowtie R_{cur} $
            \STATE $\mathbf{R^v} \leftarrow \mathbf{R^v} \setminus R_{cur}$
         \ENDIF
      \ENDFOR
   \ENDIF
\ENDWHILE

\STATE $R \leftarrow R \bowtie \overline{\Re}$
\STATE $R \leftarrow \pi_{X^{\ast}}(\sigma_{F^{\ast}}(R))$
\end{algorithmic}
\end{algorithm}

Многомерный случай аналогичен одномерному. Заметим лишь, что сперва проводится соединение набора промежуточных представлений, а только затем проверяется формула $F^{\ast}[X]$ всвязи с тем, что необходимо обеспечить полный набор указанных в формуле атрибутов.
Для алгоритма, описывающего многомерный случай, введем следующие обозначения:
$\mathbf{P} = \{P_1, \dots, P_n\} $.
$\mathbf{R'} = \{R_1', \dots, R_s'\} $.
$\{\overline{R}_1,\dots,\overline{R}_m\} = \{R^{\ast}_1,\dots,R^{\ast}_l\}
\setminus \mathbf{R'} $.
\begin{algorithm}[H]
\caption{Многомерный случай}
\begin{algorithmic}
\REQUIRE
$\mathbf{P}$,
$\mathbf{R'}$,
$X$,
$\overline{\Re} = \overline{R}_1 \bowtie \overline{R}_2\bowtie\ldots\bowtie
\overline{R}_m $,
$Y = \<\overline{\Re}> $
$F^{\ast}$,
$X^{\ast}$
\ENSURE $R = P^{\ast}$
\STATE $\mathbf{R_{pool}} \leftarrow \mathbf{R'} \cup \mathbf{P} $
\STATE $R \leftarrow P_1$
\STATE $\mathbf{R_{pool}} \leftarrow \mathbf{R_{pool}} \setminus P_1$

\WHILE {$R \nsupseteq (Y \cup \<F^{\ast}> )$}
   \FORALL{$R_{cur} \in \mathbf{R_{pool}}$}
      \IF{$R_{cur} \cap R \cap (Y \cup \<F^{\ast} > ) \neq \emptyset$}
         \STATE $ R \leftarrow R \bowtie R_{cur}$
         \STATE $\mathbf{R_{pool}} \leftarrow \mathbf{R_{pool}} \setminus
           R_{cur}$
      \ENDIF
   \ENDFOR
   
   \IF{не добавлено ни одного отношения после последнего просмотра}
      \FORALL{$R_{cur} \in \mathbf{R_{pool}}$}
         \IF{$R_{cur} \cap (Y \cup \<F^{\ast}> ) \neq \emptyset$ or $R_{cur} \in
           \mathbf{P}$}
            \STATE $ R \leftarrow R \bowtie R_i $
            \STATE $\mathbf{R_{pool}} \leftarrow \mathbf{R_{pool}} \setminus
           R_{cur}$
         \ENDIF
      \ENDFOR  
   \ENDIF
\ENDWHILE

\STATE $R \leftarrow \sigma_{F^{\ast}[X]}(R)$
\STATE $R \leftarrow R \bowtie \overline{\Re}$
\STATE $R \leftarrow \pi_{X^{\ast}}(\sigma_{F^{\ast}}(R))$
\end{algorithmic}
\end{algorithm} 


\begin{thebibliography}{9}
\bibitem{codd}
  \textit{Codd E.F., Codd S.B., Salley C.T.}
  Providing OLAP (On-line Analytical Processing) to User-Analysts: An IT
  Mandate. – Sunnyvale(CA): Codd \& Date Inc.,
  1993.
  – 31 p.
\bibitem{lecht}
  \textit{Lechtenborger J.}, \textit{Vossen G.}
  Multidimensional normal forms for data warehouse design
  // Inf. Syst.
  – 2003.
  – Vol. 28, N 5. – P. 415–434. 
\bibitem{lehner}
  \textit{Lehner W., Albrecht J., Wedekind H.}
  Normal forms for multidimensional databases
  // Proc. of the Tenth Intern. Conf. on Scientific and Statistical Database
  Management.
  – Capri, 1998.
  – P. 63–72.
\bibitem{mazon}
  \textit{Mazon J.-N., Trujillo J., Lechtenborger J.}
  Reconciling requirement-driven data warehouses with data sources via
  multidimensional normal forms
  // Data Knowledge Engineering.
  – 2007.
  – Vol. 63, N 3. – P. 725–751.
\bibitem{vassi}
  \textit{Vassiliadis P., Sellis T.}
  A survey of logical models for OLAP databases
  // SIGMOD Rec.
  – 1999.
  – Vol. 28, N 4. – P. 64–69.
\bibitem{peder}
  \textit{Pedersen T.B., Jensen C.S., Dyreson C.E.}
  A foundation for capturing and querying complex multidimensional data
  // Inf. Syst.
  – 2001.
  – Vol. 26, N 5. – P. 383–423.
\bibitem{progressive}
  \textit{Progressive ranking of range aggregates}
  / H.-G. Li, et al.
  // Data \& Knowledge Engineering.
  – 2007.
  – Vol. 63, N 1. – P. 4–25. 
\bibitem{giorg}
  \textit{Giorgini P., Rizzi S., Garzetti M.}
  Goal-oriented requirement analysis for data warehouse design
  // Proc. of the 8th ACM international Workshop on Data Warehousing and OLAP: DOLAP '05.
  – Bremen, 2005.
  – P. 47–56.

%ours  
\bibitem{zyk_pol}
  \textit{Sergey Zykin, Andrey Poluyanov}
  Multidimensional data building using Intermediate Representations.
  "Administration problems".
  2013.
  \textnumero 5. P. 54-59.

%stream
\bibitem{Olston03}
  \textit{Olston C., Jiang J., Widom J.}
  Adaptive filters for continuous queries over distributed data streams
  // Proc. of the 2003 ACM SIGMOD Intern. Conf. on Management of Data (SIGMOD '03).
  – San Diego, 2003.
  – P. 563–574.
\bibitem{Denny05}
  \textit{Denny M., Franklin M.J.}
  Predicate result range caching for continuous queries
  // Proc. of the 2005 ACM SIGMOD Intern. Conf. on Management of Data (SIGMOD '05).
  – N.-Y., 2005.
  – P. 646–657.

%logic
\bibitem{Afrati06}
  \textit{Afrati F., Li C., Mitra P.}
  Rewriting queries using views in the presence of arithmetic comparisons
  // Theoretical Computer Science.
  – 2006.
 – Vol. 368, N 1–2. – P. 88–123.

%static approach
\bibitem{baralis}
  \textit{Baralis, E., Paraboschi, S., Teniente, E.} Materialized view selection in a multidimensional database.
  // Proc. of the 23rd International Conference on Very Large Data Bases,
  Athens, Greece.
  - 1997
  - P. 318–329
\bibitem{gupta}
  \textit{Gupta, H.}
  Selection of views to materialize in a data warehouse.
  // Proceedings of the International Conference on Database Theory, Delphi.
  - Greece, 1997.
  - P. 98–112.
\bibitem{gupta-mumick}
  \textit{Gupta, H., Mumick, I.S.}
  Selection of views to materialize under a maintenance cost constraint.
  // Proceedings of the International Conference on Database Theory.
  - Israel, 1999.
  - P. 453–470.

%dynamic
\bibitem{scheuermann}
  \textit{Scheuermann, P., Shim, J., Vingralek R.}
  WATCHMAN: A data warehouse intelligent cache manager.
  // Proceedings of the 22nd International Conference on Very Large Data Bases.
  - Bombay, India, 1996.
  - P. 51–62.
\bibitem{shim}
  \textit{Shim, J., Scheuermann, P., Vingralek R.}
  Dynamic caching of query results for decision support systems.
  // Proceedings of the 11th International Conference on Scientific and
  Statistical Database Management.
  - Cleveland, OH, 1999.
  - P. 254–263.
\bibitem{kalnis}
  \textit{Kalnis, P., Papadias, D.}
  Proxy-server architectures for OLAP.
  // Proceedings of the 2001 ACM SIGMOD International Conference on Management
  of Data.
  - Santa Barbara, CA, 2001.
  - P. 367–378.
\bibitem{chang}
  \textit{Chang-Sup Park, Myoung Ho Kim b, Yoon-Joon Lee}
  Usability-based caching of query results in OLAP systems.
  // The Journal of Systems and Software
  - 2003
  - Vol. 68, - P. 103–119

%domain
\bibitem{Keller96}
  \textit{Keller M., Basu J.}
  A predicate-based caching scheme for client-server database architectures
  // VLDB Journal. – 1996. N 5. – P. 35–47.
  
  
\end{thebibliography}

\end{document}