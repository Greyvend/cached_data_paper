\author{Мосин Сергей, Зыкин Сергей}
\begin{theorem}
$P^{\ast} = \pi_{X^{\ast}} (P_{v})$
, если: 
%\textcolor{red}{!!! Селекцию можно убрать $: P^{\ast} = \pi_{X^{\ast}} (P_{v})$}
\\а) $X^{\ast} \subseteq X_{v}$
\\б) 
$\{R^{v}_{1}, \ldots, R^{v}_{s(v)}\} = \{R^{\ast}_{1}, \ldots, R^{\ast}_{l}\}$
\\в) $M (F^{\ast}) = M (F_{v}) $.
\label{th_base_eq}
\end{theorem} 
\begin{proof}
Поскольку условия теоремы являются частным случаем теоремы \ref{th_base}, то
включение $P^{\ast} \subseteq \pi_{X^{\ast}} (P_{v})$
доказано. Необходимо показать, что
$\pi_{X^{\ast}} (P_{v}) \subseteq P^{\ast}$. Пусть
кортеж $t_v \in \pi_{X^{\ast}} (P_{v})$. Покажем, что
$t_v \in P^{\ast}$. Из определения кортежа $t_v$ следует, что существует кортеж  $t' \in R^v_1 \Join R^v_2\Join\ldots \Join R^v_{s(v)}$ и $t_v = t'[X^{\ast}]$, $F_v(t')=\n{TRUE}$. По свойству операции естественного соединения $R^v_1 \Join R^v_2\Join\ldots \Join R^v_{s(v)} = R^{\ast}_1 \Join R^{\ast}_2\Join\ldots \Join R^{\ast}_l$, следовательно, $t' \in R^{\ast}_1 \Join R^{\ast}_2\Join\ldots \Join R^{\ast}_l$. В силу условия в) $F^{\ast} (t')=\n{TRUE}$. Отсюда следует, что $t_v \in P^{\ast}$. Теорема доказана.
\end{proof}
