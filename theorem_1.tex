\author{Мосин Сергей, Зыкин Сергей}
\begin{theorem}
$P^{\ast} \subseteq \pi_{X^{\ast}} (\sigma_{F^{\ast}[X_v]} (P_{v}))$
, если:
\\а) $X^{\ast} \subseteq X_{v}$
\\б) 
$ \{R^{v}_{1}, \ldots, R^{v}_{s(v)}\}
\subseteq
\{R^{\ast}_{1}, \ldots, R^{\ast}_{l}\} $
\\в) $\<F_v> \subseteq \<F^{\ast}> , M (F^{\ast}[\<F_v> ]) \subseteq
M (F_{v}) $.
\label{th_base}
\end{theorem} 
\begin{proof}
Пусть существует кортеж $t^{\ast} \in P^{\ast}$. По условию теоремы надо показать, 
что $t^{\ast} \in \pi_{X^{\ast}} (P_{v})$. Из условия $t^{\ast} \in P^{\ast}$ следует, 
что существует кортеж $t' \in R^{\ast}_1 \bowtie R^{\ast}_2\bowtie\ldots
\bowtie R^{\ast}_l$ и $t^{\ast} = t'[X^{\ast}]$. Заметим, что такой кортеж 
определяется не единственным образом. Среди возможных вариантов выберем такой, 
который удовлетворяет $F^{\ast}(t')=TRUE$. Такой кортеж существует, так как иначе
кортежа $t^{\ast}$ не было бы в $P^{\ast}$. Таким образом, существуют кортежи
$t^{\ast}_i \in R^{\ast}_i$:

\begin{equation}
t^{\ast}_i = t'[\langle R^{\ast}_i\rangle], i = 1,\dots,l,
\label{eval_eq_1}
\end{equation}
\def \intersecij {\langle R^{\ast}_i \rangle \cap \langle R^{\ast}_j \rangle}
и для любых пар $i$ и $j$, для которых $\intersecij \neq \emptyset$, выполнено:
\begin{equation}
t^{\ast}_i[\intersecij] = t'_i[\intersecij], i,j = 1,\dots,l,
\label{eval_eq_2}
\end{equation}

Поскольку условия (\ref{eval_eq_1}) и (\ref{eval_eq_2}) выполнены для всего множества
отношений $\{R^{\ast}_{1}, \ldots, R^{\ast}_{l}\}$, то они выполнены и для любого его
подмножества, в том числе и для $\{R^{v}_{1}, \ldots, R^{v}_{s(v)}\}$.
Следовательно, кортежи $t^{\ast}_i$ из отношений $R^{v}_j$ при выполнении операции
естественного соединения образуют кортеж $t''$, для которого выполнено:
$t'' = t'[\langle {\bowtie}_{i=1}^{s(v)} R^v_i \rangle]$. В силу условия а) и 
$X_v \subseteq \langle {\bowtie}_{i=1}^{s(v)} R^v_i \rangle$ получаем 
$t''[X^{\ast}] = t'[X^{\ast}] = t^{\ast}$.

Из условия  $F^{\ast}(t') = \n{TRUE}$, по утверждению
\ref{proj_property}, получаем, что проекции $F^{\ast}[\<F_v> ]$ и $F^{\ast} [X^{\ast}]$ верны на кортеже $t'$, а следовательно и на $t''$, так как формулы определены на общих атрибутах этих кортежей. Далее, по условию 3, $F_v (t'') = \n{TRUE}$.
Это значит, что $t^{\ast} = t''[X^{\ast}] \in \pi_{X^{\ast}} (P_{v})$. Теорема
доказана.
\end{proof}
