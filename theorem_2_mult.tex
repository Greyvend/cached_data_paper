\author{Мосин Сергей, Зыкин Сергей}
\begin{theorem}
$P^{\ast} = \pi_{X^{\ast}} (P_1 \Join \dots \Join 
P_n)$, где $X = \bigcupn X_{v}$ если:
\\а) $X^{\ast} \subseteq X$, $X_v \supseteq \<\Join_{i=1}^{s(v)} R^v_i> \cap (\bigcup\limits_{\substack{w=1\\ w \neq v}}^{n} \<\Join_{i=1}^{s(w)} R^w_i> ), v = 1,\dots,n$
\\б)
$ \bigcupn \{R^{v}_{1}, \ldots, R^{v}_{s(v)}\} = \{R'_{1}, \ldots, R'_{s'}\}
= \{R^{\ast}_{1}, \ldots, R^{\ast}_{l}\} $
\\в) $M(F^{\ast}) = M(F_{1} \& \dots \& F_{n})$.
\label{th_mult_eq}
\end{theorem} 
\begin{proof}

Поскольку условия теоремы являются частным случаем теоремы \ref{th_mult}, то
включение $P^{\ast} \subseteq \pi_{X^{\ast}} (\Join_{v=1}^{n} P_{v})$
доказано. Необходимо показать, что
$\pi_{X^{\ast}} (\Join_{v=1}^{n} P_{v}) \subseteq P^{\ast}$. Пусть
кортеж $t \in \pi_{X^{\ast}} (\Join_{v=1}^{n} P_{v})$. Покажем, что
$t \in P^{\ast}$. По свойству операции проекции и условию а) имеем
$\pi_{X^{\ast}}(\pi_{X_1}(R^1_1 \Join \dots \Join R^1_{s(1)}) \Join \dots
\Join \pi_{X_n}(R^n_1 \Join \dots \Join R^n_{s(n)})) = \pi_{X^{\ast}}
(R'_1 \Join \dots \Join R'_{s'})$. Таким образом, в силу определения кортежа
$t$, получаем, что существует кортеж  $t' \in R'_1 \Join \ldots \Join
R'_{s'}$ и $t = t'[X^{\ast}]$, $F_v(t')=\n{TRUE}, v = 1,\dots,n$. По свойству
операции естественного соединения $R'_1 \Join \ldots \Join R'_s = R^{\ast}_1
\Join \ldots \Join R^{\ast}_l$, следовательно, $t' \in R^{\ast}_1 \Join
\ldots \Join R^{\ast}_l$. В силу условия в) $F^{\ast} (t')=\n{TRUE}$. Отсюда следует, что $t \in P^{\ast}$.
Теорема доказана.
\end{proof} 
